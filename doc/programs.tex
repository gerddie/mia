\section{Programs}
\label{ch:prog}


This chapter lists all the software tools provided by mia2D. 
The descriptions follow a designated pattern: 

\begin{description}
\item [Program:]\emph{the command line program name}
\item [Description:]A description of the program
\item [Remarks:] additional notes
\item [Options:] list of options

\tabstart
long name & short name & value type & description of the parameter & default value\\
\hline
\tabend

\noindent 
If no value type is given, then the option accepts no further parameter. 
All programs support the following options: 

\tabstart
--help & -? & & Print out some help & \\\hline
--usage & & & Print out a short help & \\\hline
--version & -v & & Print out version information & \\\hline
--copyright & & & Print out copyright information & \\\hline
--verbose & -V & string & set the verbosity of the output out of \{ debug, message, warning, fail, error, fatal\} & error \\\hline
\tabend
\end{description}

Input from and output to a variety of 2D and 3D file formats is implemented. Note, however, only binary and gray scale data is supported.

\begin{table}[h]
\caption{\label{tab:2dformats}2D image formats}
\centering{
\begin{tabular}{|l|l|c|}
\hline
File type description & on the web & suffix \\\hline
OpenEXR HDR image format & http://www.openexr.com/ & exr \\\hline
Portable Network Graphics & http://www.libpng.org & png \\\hline
Tagged Image File format & http://www.remotesensing.org/libtiff/libtiff.html & tif \\\hline
Windows Bitmap & http://www.daubnet.com/formats/BMP.html & bmp \\\hline
\end{tabular}
}	
\end{table}

\begin{table}[h]
\caption{\label{tab:3dformats}3D image formats}
\centering{
\begin{tabular}{|l|l|c|}
\hline
File type description & on the web & suffix \\\hline
INRIA 3D File format &http://foveaproject.free.fr/softwareEng.html & inria  \\\hline
%%Medical Imaging NetCDF & http://www.bic.mni.mcgill.ca/software/minc/ & mnc \\\hline
Sun TAAC Image File Format & & vff \\\hline
Vista File Format & http://www.cs.ubc.ca/nest/lci/vista/vista.html & v \\\hline
\end{tabular}
}
\end{table}

\subsection{2D Image Conversion}

\begin{description}
\item [Program:]\emph{mia-raw2image}
\item [Description:]A raw data to 2D image image conversion program. 
\item [Options:] $\:$

\tabstart
\optinfile
\optoutfile
\opttypetwod
--big-endian & -b & bool & input data is big-endian & false \\\hline
--repn & -r & string & input pixel type (ubyte|sbyte|short|ushort|int|uint|float|double) & short \\\hline
--scale & -f & vector of float & scale of input voxels FX,FY in real world units & 1.0,1.0 \\\hline
--size & -s & vector of integer & size of input NX,NY & 1,1 \\\hline
\tabend
\item [Example:] Converting \emph{unsigned short} data in low-endian format of size $120\times130$ representing data 
  of pixel size $2mm \times 2mm$ to a tif image:
\begin{lstlisting}
mia-raw2image -i input.raw -o output.tif -r ushort -s '<120,130>' -f '<2.0,2.0>' 
\end{lstlisting}
\end{description}


\subsection{2D Image Filtering}

\begin{description}
\item [Program:]\emph{mia-2dimagefilter}
\item [Description:]A 2D image filtering program. Supported filters are described in Subsection \ref{sec:filter2d}. 
\item [Options:] $\:$

\tabstart
\optinfile
\optoutfile
\opttypetwod
\opthelpplugin
\tabend
\item [Example:] Filtering an \texttt{input.png} with a median filter and anisotropic filtering and writing the output to \texttt{output.bmp}, 
		thus converting the image to the Microsoft bitmap format after filtering. 
\begin{lstlisting}
mia-2dimagefilter -i input.png -o output.bmp -t bmp \
	median:w=3 aniso:iter=1000,psi=pm1
\end{lstlisting}
\end{description}

\begin{description}
\item [Program:]\emph{mia-2dimagefilterstack}
\item [Description:]A 2D image filtering program that filters a series of images that are numbered consecutively. 
              Supported filters are described in Subsection \ref{ch:plugins}. 
\item [Options:] $\:$

\tabstart
\optinfile
\optoutbase
\opttypetwod
\opthelpplugin
\tabend
\item [Example:] Filtering images of the pattern \texttt{input0000.png} with a median filter and anisotropic 
                 filtering and writing the output to \texttt{outputXXXX.bmp}, thus converting the image to the Microsoft 
                 bitmap format after filtering. 
		 XXXX represents the slice numbers that correspond to the input image slice numbers. 
                 The number of digits corresponds to the number of digits in the input image file names. 
\begin{lstlisting}
mia-2dimagefilterstack -i input0000.png -o output -t bmp \
        median:w=3 aniso:iter=1000,psi=pm1
\end{lstlisting}
\end{description}


\subsection{2D Non-Linear Image Registration}
\label{sec:reg2d}

\begin{description}
\item [Program:]\emph{mia-2dnrreg}
\item [Description:]voxel based 2D image registration software 
\item [Options:] $\:$

\tabstart
\optinfile
\optreffile
\optoutfile
--def-file & -d & string &  deformed inpout image & (none) \\\hline
--regmodel & -m & string & registration model  & navier \\\hline
--timestep & -t  & string & time step (fluid|direct) & fluid \\\hline
--mgsize  & -s & integer & multigrid start size & 16 \\\hline       
--max-iter & -n & maximum number of iterations & 200  \\\hline
--cost & -c & cost function & ssd             \\\hline
--interpolator & -p  & image interpolator (bspline2|bspline3|bspline4|bspline5|nn|omoms3|tri) & bspline3  \\\hline
--epsilon & -e &  relative accuracy to stop registration at a multi-grid level & 0.01  \\\hline
\tabend
\item [Example:] Registration of two images by using the naviera kernel and standart parameters:
\begin{lstlisting}
   mia-2dnrreg -i src.png -r ref.png -o src-ref-field.vf -m naviera 
\end{lstlisting}
\end{description}


\subsection{3D Image Conversion}

\begin{description}
\item [Program:]\emph{mia-raw2volume}
\item [Description:]A raw data to 3D image image conversion program. 
\item [Options:] $\:$

\tabstart
\optinfile
\optoutfile
\opttypetwod
--big-endian & -b & bool & input data is big-endian & false \\\hline
--repn & -r & string & input pixel type (ubyte|sbyte|short|ushort|int|uint|float|double) & short \\\hline
--scale & -f & vector of float & scale of input voxels FX,FY,FZ in real world units & <1.0,1.0,1.0> \\\hline
--size & -s & vector of integer & size of input NX,NY,NZ & <1,1,1> \\\hline
\tabend
\item [Example:] Converting \emph{unsigned short} data in low-endian format of size $120\times130\times64$ representing data 
  of pixel size $2mm \times 2mm \times 1mm$ to a tif image:
\begin{lstlisting}
mia-raw2image -i input.raw -o output.tif -r ushort -s '<120,130,64>' -f '<2.0,2.0,1.0>'
\end{lstlisting}
\end{description}


\begin{description}
\item [Program:]\emph{mia-3dimagefilter}
\item [Description:]A 3D image filtering program. Supported filters are described in Subsection \ref{sec:3dfilters}. 
\item [Options:] $\:$

\tabstart
\optinfile
\optoutfile
\opttypetwod
\opthelpplugin
\tabend
\item [Example:] Filtering an \texttt{input.png} with a median filter and anisotropic filtering and writing the output to \texttt{output.bmp}, 
		thus converting the image to the Microsoft bitmap format after filtering. 
\begin{lstlisting}
mia-2dimagefilter -i input.png -o output.bmp -t bmp \
	median:w=3
\end{lstlisting}
\end{description}


\subsection{3D Non-Linear Image Registration}
\label{sec:reg3d}

\begin{description}
\item [Program:]\emph{mia-2dnrreg}
\item [Description:]voxel based 2D image registration software 
\item [Options:] $\:$

\tabstart
\optinfile
\optreffile
\optoutfile
--def-file & -d & string &  deformed inpout image & (none) \\\hline
--regmodel & -m & string & registration model  & navier \\\hline
--timestep & -t  & string & time step (fluid|direct) & fluid \\\hline
--mgsize  & -s & integer & multigrid start size & 16 \\\hline       
--max-iter & -n & maximum number of iterations & 200  \\\hline
--cost & -c & cost function & ssd             \\\hline
--interpolator & -p  & image interpolator (bspline2|bspline3|bspline4|bspline5|nn|omoms3|tri) & bspline3  \\\hline
--epsilon & -e &  relative accuracy to stop registration at a multi-grid level & 0.01  \\\hline
\tabend
\item [Example:] Registration of two images by using the naviera kernel and standart parameters:
\begin{lstlisting}
   mia-3dnrreg -i src.png -r ref.png -o src-ref-field.vf -m naviera 
\end{lstlisting}
\end{description}

