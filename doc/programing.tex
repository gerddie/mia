

\begin{document}
\chapter{Programming Guide}
\lstset{numbers=left, numberstyle=\small, numbersep=5pt}

This part of the MIA guide decribes, how to use the interfaces provided to write your own softwrae, 
  how to extend the the functionallity of MIA without touchting its core, and, finally, how to 
  change MIA itself. 
\begin{itemize}
\item In chapter \ref{ch:simple} an example is provided, that describes how to scan the command line, load an image, 
  run some given filters on it and store the image. 
\item Chapter \ref{ch:images} focuses images on writing an image filter that can handle various different gray scale pixel formats.
\item Chapter \ref{ch:io} decribes how to teach MIA to read and write additional image formats. 
\item Chapter \ref{ch:filterplugin} will teach you, how to write an image filter plugin. 
\item Chapter \ref{ch:addpluginstype} gives an insight how to create a new class of plug-ins.
\end{itemize}

\section{A simple program}
\label{ch:simple}

This chapter will teach you how to use the MIA-tools to scan command line parameters, read and store images, 
  create filters opbejcts from plug-ins and apply filter chains to the images. 
In summary, the program described here is the \texttt{mia-2dimagefilter} from \texttt{src/2dimagefilter.cc}. 

For convenience the full source code is printed in the listning below, and we will go through it line by line. 
\lstset{language=c++}
\begin{lstlisting}
#include <mia/core.hh>
#include <mia/2d.hh>

NS_MIA_USE; 
using namespace std; 

int main( int argc, const char *argv[] )
{
  string in_filename;
  string out_filename;
  string out_type; 
  bool help_plugins = false; 

  try 
  {
    const C2DFilterPluginHandler::Instance& filter_plugins = 
      C2DFilterPluginHandler::instance();
  
    const C2DImageIOPluginHandler::Instance& imageio = 
      C2DImageIOPluginHandler::instance();
  
    CCmdOptionList options;  
    options.push_back(make_opt( in_filename, "in-file", 'i', 
                      "input image(s) to be filtered", "input", true)); 
    options.push_back(make_opt( out_filename, "out-file", 'o', 
                      "output image(s) that have been filtered", "output", true)); 
    options.push_back(make_opt( out_type, imageio.get_set(), "type", 't',
                      "output file type (if not given deduct from output file name)" , 
                      "image-type"));
    options.push_back(make_opt( help_plugins, "help-plugins", 0, 
                      "give some help about the filter plugins", NULL)); 

    options.parse(argc, argv); 
    vector<const char *> filter_chain = options.get_remaining(); 

    list<C2DFilterPlugin::ProductPtr> filters;
    for (vector<string>::const_iterator i = filter_chain.begin(); 
         i != filter_chain.end(); ++i) {
       C2DFilterPlugin::ProductPtr filter =  filter_plugins.produce(i->c_str()); 
       if (!filter) {
         stringstream error; 
         error << "Filter " << *i << " not found"; 
         throw invalid_argument(error.str());
       }
       filters.push_back(filter);
    }

    C2DImageIOPluginHandler::PData  in_image_list = imageio.load(in_filename);  
    if (!in_image_list || in_image_list->empty())
         throw invalid_argument(string("No images found in ")  + in_filename);

    vector<string>::const_iterator filter_name = filter_chain.begin();
    for (list<C2DFilterPlugin::ProductPtr>::const_iterator f = filters.begin(); 
         f != filters.end(); ++f) {
      for (C2DImageIOPluginHandler::Data::iterator   
         i = in_image_list->begin(); i != in_image_list->end(); ++i)
    
           *i = (*f)->filter(**i);
    }
    if ( !imageio.save(out_type, out_filename, *in_image_list) )
      throw runtime_error(string("unable to save result to ") + out_filename);
    return EXIT_SUCCESS; 

  } 
  catch (const runtime_error &e){
    cerr << argv[0] << " runtime: " << e.what() << endl;
  }
  catch (const invalid_argument &e){
    cerr << argv[0] << " error: " << e.what() << endl;
  }
  catch (const exception& e){
    cerr << argv[0] << " error: " << e.what() << endl;
  }
  catch (...){
    cerr << argv[0] << " unknown exception" << endl;
  }
  return EXIT_FAILURE;
}  
\end{lstlisting}  

\begin{description}
\item [1:] For option parsing include the core of the MIA library. 
\item [2:] For 2D image handling pull in the apropriate declaration. 
\item [4-5:] Using the MIA and STD namespace for short access. 
\item [7:] Start the main function. 
\item [9-12:] Declare some variables, that are needed for the command line parsing, namely the input and output file names, 
    the output file format (if not given the file format is deducted from the output file name), and a flag 
    see if the user requests some help on the supported filter plugins. 
    The values given to these variables will be the default values for the options. 
     As an exception, boolean values are always default to ``false''. 
\item [14-15, 64-77:] The whole code is put into a try catch block, since error handling is done by exceptions. 
\item [16-20:] Define some short cuts for the plugin handlers.
\item [22-31:] Define the options for this program using the variables given above. 
    Silently, in the background are some more options available, that deal with common tasks, such as printing out the general
      help, printing out copyright information, and verbosity level of the output. 
    See chapter \ref{ch:prog} for details. 
\item [33-34:] Parse the command line. 
     For this program, the remaining command line parameters describe the filters to be applied. 
    A command line like 
    \lstset{language=bash,numbers=none}
    \begin{lstlisting}
eva-2dimagefilter -i input.png -o output.png \
    downscale:bx=2,by=2 bandpass:min=20,max=200
    \end{lstlisting}
      will be parsed like follows:
      \begin{itemize}
      \item  \texttt{in\_filename} = ``input.png''
      \item  \texttt{out\_filename} = ``output.png''
      \item  \texttt{filter\_chain} = (downscale:bx=2,by=2, bandpass:min=20,max=200)
      \end{itemize}

\item [36-46:] Create the filter chain by using the filter descriptions from the command line. 
       If the creation of one filter fails, an \texttt{invalid\_argument} exception is thrown, which will terminate the 
       program with an error message. 
\item [48-50:] Load the image(s) from the input file. 
       If no input image is found, throw an \texttt{invalid\_argument} exception that will terminate the program with an error message. 
\item [52-59:] Apply the filters to all input images replacing the input images. 
       Since the images are all wrapped into shared pointers, no memory leaks will be introduced by this code. 
\item [60-61:] Save the filtered image(s) in the output file. If not successfull throw an exception.
\item [62:] Terminate the program with the return value indicating success.
\item [64-77:] Catch all known and unknown exceptions and report.
\item [78:] return signalling failure, because somewhere in the program a exception was thrown that could only be reported 
    in the main function. 
\end{description}

\section{Images and Filtering}

\label{ch:images}

In the following, the basic handling of images is described. 
All the examples will use 2D images. 
If not otherwise noted, 3D images are handled likewise. 

\section{How to create and copy images of a certain image type}

The images supported in MIA  may contain different pixel types, e.g. 1-bit, 8-bit, 16-bit, or even float valued ones. 
MIA uses shared pointers to \texttt{C2DImage}, respectively,  to hold images. 
However, in order to account for the pixel type, at creation time a derivative class of the used pixel type needs to be specified. 
In order to create an 16-bit image with unsigned pixel values with a given \texttt{size}, use:

\begin{lstlisting}
P2DImage image = P2DImage(new C2DUSImage(size)); 
\end{lstlisting}
 
In order to copy such an image, an image copy filter is provided, that can be invoked like follows: 

\begin{lstlisting}
P2DImage image_copy = filter<T2DImage>(FCopyImage(), image); 
\end{lstlisting}
\emph{Remark: a clone method might come in handy ...}

More about these image filters is described in the next section: 

\section{An Image Filter}
\label{sec:filter}

In order to access the image data, its type needs to be known. 
Of course it is possible to use the \texttt{get\_type} method of the \texttt{C2DImage} class and use
  a switch statement in order to cast to the appropriate derived class. 
However, since this needs to be done very often, MIA provides various template functions to handle the most cases
  of image access. 
Namely, functions \texttt{mia::filter} are provided that take a filter functor as argument as well as one image or two images. 
Examples of its usage follow: 
Imagine, an image thresh-holding filter, that takes an image, and sets all pixels above or below given thresholds to zero, and 
  all pixels within the range to one. 
Such a filter would be defined like follows:     

\lstset{language=c++,numbers=left}	
\begin{lstlisting}
#include <mona/core/filter.hh>

struct FThreshold: public TFilter<P2DImage> {
  FThreshold(double min, double max):
    _M_min(min), 
    _M_max(max)
    {
    }
  typename <template T> 
  FThreshold::result_type operator()(const T2DImage<T>& image)const {
    typename T2DImage<T>::const_iterator ii = image.begin(); 
    typename T2DImage<T>::const_iterator ei = image.end();
    
    C2DBitImage *r = new  C2DBitImage(image.get_size(), image.get_attribute_list()); 
    P2DImage result(r); 
    C2DBitImage::iterator ir = r->begin();
    
    while (ii != ei) {
      *ir++ = (*ii >= _M_min && *ii < _M_max); 
      ++ii; 
    }
    return result;     
  }
private: 
  double _M_min; 
  double _M_max; 
); 
\end{lstlisting}

Here, \texttt{TFilter} is a template, that defines the type \texttt{return\_type} that is needed for the call to the actual filter function. 
The constructor of the filter functor \texttt{FThreshold} initialises the filter with the given parameters. 
A \texttt{do\_threshold} function that makes use of this filter operator would look like this:   
  
\begin{lstlisting}
P2DImage do_threshold(double min, double max, const C2DImage& image)
{  
  return  mia::filter<T2DImage>(FThreshold(min,max), image);
}   
\end{lstlisting}

The \texttt{filter<T2DImage>} function takes care of casting the input image depending on its pixel type, and then invokes 
  the operator () of \texttt{FThreshold} and returns its result.
  
\section{Pixel Type Dependant Filtering}
\label{sec:ptdf}

Imaging the situation, when a filter type is only appropriate for certain pixel types, or should behave differently for different
   pixel types. 
Then using template specification comes to the rescue. 
Imagine, for example, in the above example, using the thresh-holding filter on a bit-valued image doesn't make much sense. 
Therefore, we would like to report an error, if the filter is invoked with an bit valued image. 
This can be done  by additionally implementing a specialisation of the operator () of \texttt{FThreshold}: 
\emph{I think another indirection is needed here ...}
\begin{lstlisting}
template <>
FTreshold::result_type FTreshold::operator(const C2DBitImage& image)const 
{
  throw invalid_argument("FTreshold can not be used on bit-valued images"); 
}
\end{lstlisting}
 
The compiler will take care of the rest, and if the user provides a bit valued image the above exception will be raised. 


\section{Extending MIA}
\label{ch:io}

\subsection{Adding a Plugin to store/load data in a new file format}
\label{sec:newfilehandler}

In the following, adding support fo a new 2D image file format is described. 
The base class defining the interface for 2D Image IO is \texttt{C2DImageIOPlugin} and it is 
  declared in the the header file \texttt{<mia/2d/2dimageio.hh>}. 
In order to implement the new IO file format, four abstract virtual functions have to be implemented: 
\begin{itemize}  
\item {\tt do\_add\_suffixes}: Adds the standard suffixes for this file format to the plugin handlers suffic map. 
\item {\tt do\_load}: Implements loading data from the file. 
\item {\tt do\_save}: Implements storing data to the file. 
\item {\tt do\_get\_descr}: Returns a string with a short description of the plugin.
\end{itemize}  
In addition, a constructor has to be defined, in order to name the format. 

The following example code is taken from the Windows BMP iamge format IO plugin which resides in the MIA source tree in \texttt{mia/2d/io/bmp.cc}.
First, the plugin class is declared as a specification of the general 2D plugin class: 

\begin{lstlisting}{BMPIO}
#include <mia/2d/2dimageio.hh>
NS_BEGIN(BMPIO)
NS_MIA_USE

class CBMP2DImageIO: public C2DImageIOPlugin {
public:
  CBMP2DImageIO();
private: 
  void do_add_suffixes(multimap<string, string>& map) const;
  PData do_load(const string& fname) const;
  bool do_save(const string& fname, const Data& data) const;
  const string do_get_descr() const; 
};
\end{lstlisting}

\noindent 
The constructor of the plugin takes care of two things: initialise the plug-in with its apropriate name and set the supported pixel formats. 

\begin{lstlisting}{BMPIO}
CBMP2DImageIO::CBMP2DImageIO():
	C2DImageIOPlugin("bmp")
{
	add_supported_type(it_ushort);
	add_supported_type(it_ubyte); 
	add_supported_type(it_bit);
}
\end{lstlisting}

\noindent 
Since at construction time, nothing is known about the plugin handler, the setup of the suffix-fileformat map has to be done in an extra function: 
\begin{lstlisting}{BMPIO}
void CBMP2DImageIO::do_add_suffixes(multimap<string, string>& map) const
{
	map.insert(pair<string,string>(".bmp", get_name())); 
	map.insert(pair<string,string>(".BMP", get_name())); 
}
\end{lstlisting}


\noindent 
The next function just gives a plain text description of the plug-in, that might be used wehn printing out help. 
\begin{lstlisting}{BMPIO}
const string  CBMP2DImageIO::do_get_descr()const
{
	return string("BMP 2D-image input/output support");
}
\end{lstlisting}

\noindent 
Since some file types (like, e.g. TIFF) provide their own routines for opening the file, this filr oprn code needs to be added here.
After creating the result image vector, format specific code follows to load the image data that is then returned. 
\begin{lstlisting}{BMPIO}
CBMP2DImageIO::PData CBMP2DImageIO::do_load(string const& filename)const
{
  int read = 0; 
  CInputFile f(filename);
  if (!f)
    return PData(); 

  PData result = PData(new C2DImageVector()); 

 
  // File format specific implementation follows
  //  ...
	
  return result; 
}
\end{lstlisting}

\noindent 
In the save function, again the file needts to be opened/created for output, because this might by format specific. 
Saving the image data requires a translation of the pixel type that is best provided by a filter function (see Section \ref{sec:filter}) here 
  given by the CImageSaver class.
\begin{lstlisting}{BMPIO}
bool CBMP2DImageIO::do_save(string const& filename, const C2DImageVector& data) const
{
	cvdebug() << "CBMP2DImageIO::save begin\n"; 
	
	COutputFile f(filename);
	if (!f) {
		cverr() << "CBMP2DImageIO::save:unable to open output file:" << filename << "\n"; 
		return false; 
	}
		
	CImageSaver saver(f); 
	
	for (C2DImageVector::const_iterator iimg = data.begin(); iimg != data.end(); ++iimg)
		filter(saver, **iimg); 
	
	cvdebug() << "CBMP2DImageIO::save end\n"; 
	return true; 
}
\end{lstlisting}

\noindent 
Finally, the plugin interface function needs to be defined. 
This functions needs to be declared \texttt{extern ``C''} to avoid the C++ name mangeling that is compiler dependend. 
Depending on the target platform and compiler, some additional decoration might be necessary that is set elsewhere 
  through the \texttt{EXPORT} define.
\begin{lstlisting}{BMPIO}
extern "C" EXPORT CPluginBase *get_plugin_interface()
{
		return new CBMP2DImageIO;
}
\end{lstlisting}

\noindent 
Above code skeleton provides the basis for all IO plugins. 
For further reading, the source code in the respective ``io'' directories might hold forward. 

\subsection{Writing an Image Filter Plugin}
\label{ch:filterplugin}

\subsection{Adding a new general plugin type}
\label{ch:addpluginstype}

\subsection{Adding new data types}
\label{sec:adddatatypeio}


\bibliographystyle{plainnat}
\cleardoublepage\addcontentsline{toc}{chapter}{\bibname}
\bibliography{userguide}
\end{document}