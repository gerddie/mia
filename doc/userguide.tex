
\documentclass[english, 10pt, a4paper,headsepline,openany]{scrbook}
\usepackage[T1]{fontenc}
\usepackage[latin1]{inputenc}
\usepackage[left=2cm,right=1cm,top=2cm,bottom=2cm,twoside]{geometry}
\usepackage{array}
\usepackage{amsmath}
\usepackage{amssymb}
\usepackage[numbers]{natbib}
\usepackage{listings}
\usepackage{color}
\usepackage{graphics}
\usepackage{nonfloat}
\usepackage{babel}
\usepackage[colorlinks,linkcolor=blue]{hyperref}
\usepackage{multirow}
\usepackage{ifthen}
\usepackage{tabularx}
\usepackage{rotating}
\usepackage{colortbl}
\newcommand{\has}{\cellcolor[gray]{0.5}}

\include{version}
\newcommand{\vx}{\ensuremath{\mathbf{x}} }
\newcommand{\vv}{\ensuremath{\mathbf{x}} }
\newcommand{\vu}{\ensuremath{\mathbf{u}} }
\newcommand{\n}{\ensuremath{\mathbf{n}} }
\newcommand{\R}{\ensuremath{\mathbb{R}} }
\newcommand{\curl}[1]{\nabla \times #1} % for curl

\newcommand\tabstart{\begin{tabular}{|p{0.12\columnwidth}|
					     p{0.04\columnwidth}|
                                             p{0.1\columnwidth}|
                                             l|}
                          \hline}
\newcommand\tabend{\end{tabular}}

\newcommand\cmdopt[3]{#1 & #2 & #3 \\\hline}
\newcommand\optinfile{--in-file  & -i & string & input file name  \\\hline}
\newcommand\optreffile{--ref-file  & -r & string & reference file name  \\\hline}
\newcommand\optoutfile{--out-file & -o & string & output file name  \\\hline}
\newcommand\optoutbase{--out-base & -o & string & output file name base - a number and the type name will be added automatically \\\hline}
\newcommand\opttypetwod{--type      & -t & string & output image file type (see Table \ref{tab:2dformats})  \\\hline}
\newcommand\opttypetwodb{--type      & -t & string & output image file type (see Table \ref{tab:2dformats})  \\\hline}
\newcommand\opttypethreed{--type      & -t & string & output image file type (see Section \ref{tab:3dformats})  \\\hline}
\newcommand\opthelpplugin{--help-plugins & & & print out all plugins and a short description \\\hline}

\newcommand\plugtabstart{\begin{tabular}{|p{0.1\columnwidth}|p{0.1\columnwidth}|p{0.5\columnwidth}|p{0.1\columnwidth}|}\hline
Name & Type & Description & Default \\\hline\hline} 
\newcommand\plugtabend{\hline\end{tabular}}

\newcommand\miamex{{\tt miamex} }
\newcommand{\tw}[1]{{\tt #1}}


%\makeatletter
\definecolor{listinggray}{gray}{0.9}
\lstset{backgroundcolor=\color{listinggray}}


\makeatother

\begin{document}

\vfill{}
\title{MIA User Guide \\Software Version: \miaversion}
\vfill{}


\author{Gert Wollny}

\maketitle

\section*{Preface}

This is the \emph{MIA User Reference}. 
In this document you will find descriptions for all the programs and plug-ins currently included in the 
  MIA software package. 
This document is maintained by Gert Wollny <gw.fossdev@gmail.com>. 
Additions, hints, modifications, and corrections are welcome. 

\section*{License}

Copyright (c) Leipzig, Madrid 2004-2011, Gert Wollny <gw.fossdev@gmail.com>
Permission is granted to copy, distribute and/or modify this document under the terms of the 
  GNU Free Documentation License, Version 1.1 or any later version published by the 
  Free Software Foundation; with no Invariant Sections, 
  with no Front-Cover Texts and with no Back-Cover Texts. 
A copy of the license is available at http://www.gnu.org/copyleft/fdl.html

\section*{Changes}

\begin{center}
\begin{tabular}{cc}
\hline 
Date  & Description\\
\hline
\hline 
01/08/2011 & Second Public Release  \\
01/10/2007 & First Public Release  \\
01/03/2007 & Second Beta \\
13/06/2007 & First Beta  \\
\hline
\end{tabular}
\end{center}

\tableofcontents{}

\pagestyle{headings}


\chapter{Introduction}

MIA is a library and a collection of command line programs for gray scale image processing. 
It's main focus is on research in image processing.
For that reason some of the programs are of a rather experimental nature and partially 
   duplicate funcionality. 
Nevertheless, the funcionality that is provided by the libraries and plug-ins is developed 
  using \emph{test driven development} to ensure reproducability of the results.
 


\section{Installation}

MIA has been tested on GNU/Linux (x86,amd64) but should be installable on any Posix compatible platform. 


\subsection{UNIX, Linux, Mac OS X} 

In order to use MIA, on these platforms the software is best installed from source code. 
To do so, your software environment has to meet the following pre-requisites: 

\begin{enumerate}
\item You need a ANSI-compatible C++ compiler that supports some features of the new C++0x standard 
       (most likely dubbed C++2011) -- GNU g++ (>=4.5) \url{http://gcc.gnu.org} is known to work. 
\item Intel Threading Building Blocks for Open Source \url{http://threadingbuildingblocks.org/}
\item CMake \url{http://www.cmake.org}
\item The BOOST >= 1.40 \url{http://www.boost.org} library
\item gsl The GNU Scientific Library \url{http://www.gnu.org/software/gsl/}
\item A library that provides the CBLAS interface.
\item The Intel Threading Building Blocks \url{http://threadingbuildingblocks.org/}
\end{enumerate}

\noindent 
Additionally, packages listed in Table \ref{tab:external} add to the functionallity of the software, 
   if available and enabled.

\begin{table}[h]
\caption{\label{tab:external}Supported external packages}
\begin{tabularx}{\linewidth}{lX}
\hline 
{\bf Package } & {\bf Additional information} \\
\hline 
\hline 
{ OpenEXR } & A HDR Image Library that supports 32 bit and floating point valued 
                 images \url{http://www.openexr.org} \\
\hline 
{ TIFF } & The Tagged Image File Format \url{http://www.remotesensing.org/libtiff/libtiff.html} \\
\hline 
{ PNG } & Portable Network Graphics \url{http://www.libpng.org} \\
\hline 
{ DCMTK } & DICOM image IO (partial support) \url{http://dicom.offis.de/dcmtk} \\
\hline 
{ nlopt } & Nonlinear optimizers library  \url{http://ab-initio.mit.edu/wiki/index.php/NLopt} \\
\hline 
{ it++ } & Signal processing library \url{http://itpp.sourceforge.net}  \\
\hline 
{ xml++ } & Input/output of XML files, support for certain segmentation features \url{http://libxmlplusplus.sourceforge.net/} \\
\hline 
{ fftw } & Fast Fourier Transformation \url{http://www.fftw.org}  \\
\hline 
\end{tabularx}
\end{table}

If all of the above pre-requisites are met, the installation of MIA boils down to:

\lstset{language=bash}
\begin{lstlisting}
  tar -zxvf mia-2.0.X.tgz
  mkdir build-mia
  cd build-mia
  cmake -DCMAKE_INSTALL_PREFIX=<path to install> ../mia-2.0.X
  make
  make install
\end{lstlisting}
In order to test some of the components, you may run ``make test''. 

\subsection{MS Windows} 

Currently broken, may work with cygwin. 

\chapter{User Reference}


In its essence, MIA is a collection of command line programs and supporting plug-ins. 


\section{Image and pixel formats}

The MIA library operates exclusively on gray scale and binary images. 
The supported pixel formats are given in Table \ref{tab:pixform}. 

\begin{table}[h]
\caption{\label{tab:pixform}Supported pixel formats}
\begin{tabular}{lrr}
Short name & C data type & size in bit \\
\hline 
\hline 
bit    & bool          & 1  \\
\hline 
sbyte  & signed char   & 8  \\ 
\hline 
ubyte  & unsigned char & 8  \\ 
\hline 
sshort & signed short  & 16 \\
\hline 
ushort & unsigned short& 16 \\
\hline 
sint   & signed int    & 32 \\
\hline 
uint   & unsigned int  & 32 \\
\hline 
float  & float         & 32 \\
\hline 
double & double        & 64 \\
\hline 
\end{tabular}
\end{table}

The supported file formats depend, in part, on the external libraries that were enabled during build time. 
Table \ref{tab:fileformats} lists all file formats can can be enabled. 

\begin{table}[h]
\caption{\label{tab:fileformats}Supported image file formats}
\begin{tabular}{lll|c|c|c|c|c|c|c|c|c|c}
          &     &  & \multicolumn{9}{c|}{\bf Supported pixel types} & \\

{\bf Format}    &  {\bf Extension }    &  {\bf Depends} & \ventry{bit} & \ventry{sbyte} & \ventry{ubyte} & \ventry{sshort} 
                                             & \ventry{ushort} & \ventry{sint} & \ventry{uint} & \ventry{float} 
                                             & \ventry{double} & \ventry{multi-}-\ventry{record} \\
\hline 
\hline 
\multicolumn{3}{l}{\bf 2D image files types} & \multicolumn{9}{l}{} & \\
\hline 
Windows bitmap            & bmp &            & \has &    &\has &    &\has &    &    &   &   &   \\
\hline 
Internal storage          & @   &            &\has& \has&\has &\has &\has &\has &\has &\has&\has&   \\
\hline 
DICOM                     & dcm & dcmtk      &   &    &    &\has &    &    &    &   &   &   \\
\hline 
OpenExr                   & exr & openexr    &   &    &    &    &    &    &\has &\has&   &\has\\
\hline 
Portable Network Graphics & png & libpng     &\has&    &\has &    &\has &    &    &   &   &   \\
\hline 
Raw data (output)         & raw &            &\has&\has &\has &\has &\has &\has &\has &\has&\has&   \\
\hline 
Tagged Image File Format  & tif, tiff & tiff &\has&    &\has &    &\has &    &    &   &   &\has\\
\hline 
Vista file format         & v   &            &\has&\has &\has &\has &    &\has &    &\has&\has&\has\\
\hline 
\multicolumn{3}{l}{\bf 3D image files types} & \multicolumn{9}{l}{} &\\
\hline 
Analyze                   & hdr+img  &       &   &    &\has &\has &    &\has &    &\has&\has&\has\\
\hline 
DICOM (as series of files)& dcm & dcmtk      &   &    &    &\has &    &    &    &   &   &   \\
\hline 
INRIA 3D File format      &     &            &   &\has &\has &\has &\has &\has &\has &\has&\has&   \\
\hline 
Internal storage          & @   &            &\has& \has&\has &\has &\has &\has &\has &\has&\has&   \\
\hline 
Sun TAAC Image File Format& vff &            &   &    &\has &\has &    &    &    &   &   &   \\
\hline 
Vista file format         & v   &            &\has&\has &\has &\has &    &\has &    &\has&\has&\has\\
\hline 
\end{tabular}
\end{table}


%\section{Command Line Tools}

This part of the user guide fucuses on using the command line interface of the MIA programs. 
Although it is possible, to run the programs from the MS Windows command interpreter, installing 
  a UNIX-like environment like Cygwin or MingW is highly advised. 
The examples given below will all use the Bash shell script syntax. 

\subsection{Converting raw data to images and volumes}

MIA doen't support the processing of raw data files. 
Instead it provides the tools to convert such data to certain file formats that carry size and data type information. 
Two command are provided: \texttt{mia-raw2image} for conversion of 2D data and \texttt{mia-raw2volume} for conversion of 3D data. 
Some storage formats also may provide information about pixel/voxel size. 

To convert a raw 2D array of size 120x130 of double floating point values (64 bit) , stored as big endian, and 
  with a pixel size of $2.0 x 2.0$ units to a PNG image, the following command may be issued: 
\lstset{language=bash}
\begin{lstlisting}
mia-raw2image -i input.raw --big-endian -o output.png -r double \
	-s <120,130> -f <2.0,2.0> 
\end{lstlisting}

\noindent 
In order to convert a raw 256x256x128 3D data array of unsingned short values (16 bit), given as low endian, with a voxel size of $1.0 x 1.0 x 2.0$ 
  to the VISTA file format issue the following command: 
\begin{lstlisting}
mia-raw2volume -i input.raw  -o output.v -r ushort \
	-s <256,256,128> -f <1.0,1.0,2.0>
\end{lstlisting}

\noindent 
The number of supported image file formats depends on the compiled in support - see chapter \ref{ch:prog} for details. 

\subsection{Filtering images}

MIA provides a set of standard 2D and 3D filters. 
These filters can easily combined  to a filter chain. 
For example in order to filter a 2D image \texttt{image.png} first with a Gaussian of the order 3 (filter width = 2 * 3 + 1), and then with a 
  median least square filter of the order 2, issue the command 
\begin{lstlisting}
mia-2dimagefilter -i image.png -o filtered.png gauss:w=3 mlv:w=2
\end{lstlisting}

Morphological operations, line erosion, dilation, open, and close are currently only supported for 3D images. 
To apply an openin operation to a volume data set \texttt{volume.v}, by using a filled sphere of radius 3 as structuring element, 
  call: 
\begin{lstlisting}
mia-2dimagefilter -i volume.v -o morphed.v open:shape=[sphere:r=3]
\end{lstlisting}

The available filters and structuring elements are described in chapter \ref{ch:plugins}. 
Given the plug-in based architecture, more filters can easyly be added (see programming guide).

\subsection{Non-linear image registration and deformation}

MIA implements fluid dynamic and linear elastic registration by minimising a voxel based measure. 
As a result a vector field is evaluated that describes a probable course of change the given input images and the appllied registration model. 

For a non-linear fluid dynamics based registration of the floating image \texttt{source.v} to the reference image \texttt{reference.v}, by using the 
 adaptive solver algorithm \cite{wollny02comput} with a maximum of 40 iterations and a multi-grid start size of 16x16x16, call 

\begin{lstlisting}
mia-3dnrreg -i source.v -r reference.v -m naviera:iter=40 -m 16 -o field.v 
\end{lstlisting}

\noindent 
In order to apply the deformation to an input image, using the omoms3 interpolation filter, do 
\begin{lstlisting}
mia-3ddeform -i source.v -t field.v -o deformed.v -p omoms3
\end{lstlisting}







\chapter{Programs}
\label{ch:prog}

\lstset{language={}}

This chapter lists all the software tools provided by mia. 
The descriptions follow a designated pattern: 

\subsection*{The command line program name}
\begin{description}
\item [Description:]A description of the program
\item [Options:] list of options

\noindent 
\optiontable{
long name & short name & value type & description of the parameter \\
\hline
}

\noindent 
If no value type is given, then the option is a flag that defaults to \emph{false} and accepts no 
     further value. 
All programs support the following options: 

\optiontable{
\cmdopt{help}{?}{}{Print out some help}
\cmdopt{usage}{}{}{Print out a short help}
\cmdopt{version}{v}{}{Print out version information}
\cmdopt{copyright}{}{}{Print out copyright information}
\cmdopt{verbose}{V}{string}{set the verbosity of the output out of (debug, message, warning, fail, error, fatal)}
}

\item [Example:] Example how to run the program 

\item [Remarks:] additional notes
\item [See also:] Cross reference to related programs. 
\end{description}

For all programs, the \texttt{-{}-help} option will print out additional information, like which parameters 
  are required and what are the default values (if set).
Note, when other options are set alongside the \texttt{-{}-help} option, then for these options, the given value 
  will be shown as default and the \emph{required} tag will not be shown.
When a program is called with two (ore more) conflicting options then one option takes automatically precedence 
  as it is documented. 

\subsection*{An additional note for setting command line options}

Option values, that belong to one option \emph{must not} contain whitespace characters, otherwise, 
  the command line parser will seperate the parts. 
As an alternative, the command line parameter may be protected by quotes or double quotes. 
Note, however, when you give free parameters by using ``?'' or ``*'' to use the shell expansion to indicate 
  more then one file, quoting the parameter will inhibit parameter expansion, and internally, all programs 
  take the letters ``?'' and ``*'' literal. 
Also note, that some options require the use of the charaters ``<'' and ``>''. 
Since these letters are also interpreted by the shell, the parameters using these characters must be 
  protected by quotes. 

\chapter{Programs}
\label{ch:prog}


This chapter lists all the software tools provided by mia2D. 
The descriptions follow a designated pattern: 

\begin{description}
\item [Program:]\emph{the command line program name}
\item [Description:]A description of the program
\item [Remarks:] additional notes
\item [Options:] list of options

\tabstart
long name & short name & value type & description of the parameter & default value\\
\hline
\tabend

\noindent 
If no value type is given, then the option accepts no further parameter. 
All programs support the following options: 

\tabstart
--help & -? & & Print out some help & \\\hline
--usage & & & Print out a short help & \\\hline
--version & -v & & Print out version information & \\\hline
--copyright & & & Print out copyright information & \\\hline
--verbose & -V & string & set the verbosity of the output out of \{ debug, message, warning, fail, error, fatal\} & error \\\hline
\tabend
\end{description}

Input from and output to a variety of 2D and 3D file formats is implemented. Note, however, only binary and gray scale data is supported.

\begin{table}[h]
\caption{\label{tab:2dformats}2D image formats}
\centering{
\begin{tabular}{|l|l|c|}
\hline
File type description & on the web & suffix \\\hline
OpenEXR HDR image format & http://www.openexr.com/ & exr \\\hline
Portable Network Graphics & http://www.libpng.org & png \\\hline
Tagged Image File format & http://www.remotesensing.org/libtiff/libtiff.html & tif \\\hline
Windows Bitmap & http://www.daubnet.com/formats/BMP.html & bmp \\\hline
\end{tabular}
}	
\end{table}

\begin{table}[h]
\caption{\label{tab:3dformats}3D image formats}
\centering{
\begin{tabular}{|l|l|c|}
\hline
File type description & on the web & suffix \\\hline
INRIA 3D File format &http://foveaproject.free.fr/softwareEng.html & inria  \\\hline
%%Medical Imaging NetCDF & http://www.bic.mni.mcgill.ca/software/minc/ & mnc \\\hline
Sun TAAC Image File Format & & vff \\\hline
Vista File Format & http://www.cs.ubc.ca/nest/lci/vista/vista.html & v \\\hline
\end{tabular}
}
\end{table}

\section{2D Image Conversion}

\begin{description}
\item [Program:]\emph{mia-raw2image}
\item [Description:]A raw data to 2D image image conversion program. 
\item [Options:] $\:$

\tabstart
\optinfile
\optoutfile
\opttypetwod
--big-endian & -b & bool & input data is big-endian & false \\\hline
--repn & -r & string & input pixel type (ubyte|sbyte|short|ushort|int|uint|float|double) & short \\\hline
--scale & -f & vector of float & scale of input voxels FX,FY in real world units & 1.0,1.0 \\\hline
--size & -s & vector of integer & size of input NX,NY & 1,1 \\\hline
\tabend
\item [Example:] Converting \emph{unsigned short} data in low-endian format of size $120\times130$ representing data 
  of pixel size $2mm \times 2mm$ to a tif image:
\begin{lstlisting}
mia-raw2image -i input.raw -o output.tif -r ushort -s '<120,130>' -f '<2.0,2.0>' 
\end{lstlisting}
\end{description}


\section{2D Image Filtering}

\begin{description}
\item [Program:]\emph{mia-2dimagefilter}
\item [Description:]A 2D image filtering program. Supported filters are described in Subsection \ref{sec:filter2d}. 
\item [Options:] $\:$

\tabstart
\optinfile
\optoutfile
\opttypetwod
\opthelpplugin
\tabend
\item [Example:] Filtering an \texttt{input.png} with a median filter and anisotropic filtering and writing the output to \texttt{output.bmp}, 
		thus converting the image to the Microsoft bitmap format after filtering. 
\begin{lstlisting}
mia-2dimagefilter -i input.png -o output.bmp -t bmp \
	median:w=3 aniso:iter=1000,psi=pm1
\end{lstlisting}
\end{description}

\begin{description}
\item [Program:]\emph{mia-2dimagefilterstack}
\item [Description:]A 2D image filtering program that filters a series of images that are numbered consecutively. 
              Supported filters are described in Subsection \ref{ch:plugins}. 
\item [Options:] $\:$

\tabstart
\optinfile
\optoutbase
\opttypetwod
\opthelpplugin
\tabend
\item [Example:] Filtering images of the pattern \texttt{input0000.png} with a median filter and anisotropic 
                 filtering and writing the output to \texttt{outputXXXX.bmp}, thus converting the image to the Microsoft 
                 bitmap format after filtering. 
		 XXXX represents the slice numbers that correspond to the input image slice numbers. 
                 The number of digits corresponds to the number of digits in the input image file names. 
\begin{lstlisting}
mia-2dimagefilterstack -i input0000.png -o output -t bmp \
        median:w=3 aniso:iter=1000,psi=pm1
\end{lstlisting}
\end{description}


\section{2D Non-Linear Image Registration}
\label{sec:reg2d}

\begin{description}
\item [Program:]\emph{mia-2dnrreg}
\item [Description:]voxel based 2D image registration software 
\item [Options:] $\:$

\tabstart
\optinfile
\optreffile
\optoutfile
--def-file & -d & string &  deformed inpout image & (none) \\\hline
--regmodel & -m & string & registration model  & navier \\\hline
--timestep & -t  & string & time step (fluid|direct) & fluid \\\hline
--mgsize  & -s & integer & multigrid start size & 16 \\\hline       
--max-iter & -n & maximum number of iterations & 200  \\\hline
--cost & -c & cost function & ssd             \\\hline
--interpolator & -p  & image interpolator (bspline2|bspline3|bspline4|bspline5|nn|omoms3|tri) & bspline3  \\\hline
--epsilon & -e &  relative accuracy to stop registration at a multi-grid level & 0.01  \\\hline
\tabend
\item [Example:] Registration of two images by using the naviera kernel and standart parameters:
\begin{lstlisting}
   mia-2dnrreg -i src.png -r ref.png -o src-ref-field.vf -m naviera 
\end{lstlisting}
\end{description}


\section{3D Image Conversion}

\begin{description}
\item [Program:]\emph{mia-raw2volume}
\item [Description:]A raw data to 3D image image conversion program. 
\item [Options:] $\:$

\tabstart
\optinfile
\optoutfile
\opttypetwod
--big-endian & -b & bool & input data is big-endian & false \\\hline
--repn & -r & string & input pixel type (ubyte|sbyte|short|ushort|int|uint|float|double) & short \\\hline
--scale & -f & vector of float & scale of input voxels FX,FY,FZ in real world units & <1.0,1.0,1.0> \\\hline
--size & -s & vector of integer & size of input NX,NY,NZ & <1,1,1> \\\hline
\tabend
\item [Example:] Converting \emph{unsigned short} data in low-endian format of size $120\times130\times64$ representing data 
  of pixel size $2mm \times 2mm \times 1mm$ to a tif image:
\begin{lstlisting}
mia-raw2image -i input.raw -o output.tif -r ushort -s '<120,130,64>' -f '<2.0,2.0,1.0>'
\end{lstlisting}
\end{description}


\begin{description}
\item [Program:]\emph{mia-3dimagefilter}
\item [Description:]A 3D image filtering program. Supported filters are described in Subsection \ref{sec:3dfilters}. 
\item [Options:] $\:$

\tabstart
\optinfile
\optoutfile
\opttypetwod
\opthelpplugin
\tabend
\item [Example:] Filtering an \texttt{input.png} with a median filter and anisotropic filtering and writing the output to \texttt{output.bmp}, 
		thus converting the image to the Microsoft bitmap format after filtering. 
\begin{lstlisting}
mia-2dimagefilter -i input.png -o output.bmp -t bmp \
	median:w=3
\end{lstlisting}
\end{description}


\section{3D Non-Linear Image Registration}
\label{sec:reg3d}

\begin{description}
\item [Program:]\emph{mia-2dnrreg}
\item [Description:]voxel based 2D image registration software 
\item [Options:] $\:$

\tabstart
\optinfile
\optreffile
\optoutfile
--def-file & -d & string &  deformed inpout image & (none) \\\hline
--regmodel & -m & string & registration model  & navier \\\hline
--timestep & -t  & string & time step (fluid|direct) & fluid \\\hline
--mgsize  & -s & integer & multigrid start size & 16 \\\hline       
--max-iter & -n & maximum number of iterations & 200  \\\hline
--cost & -c & cost function & ssd             \\\hline
--interpolator & -p  & image interpolator (bspline2|bspline3|bspline4|bspline5|nn|omoms3|tri) & bspline3  \\\hline
--epsilon & -e &  relative accuracy to stop registration at a multi-grid level & 0.01  \\\hline
\tabend
\item [Example:] Registration of two images by using the naviera kernel and standart parameters:
\begin{lstlisting}
   mia-3dnrreg -i src.png -r ref.png -o src-ref-field.vf -m naviera 
\end{lstlisting}
\end{description}



\chapter{Plug-ins}
\label{ch:plugins}

In this chapter the plug-ins are described that are provided by the library.
\section{List of Filters and associated plug-ins}
\label{ch:plugins}


This chapter lists all the plugins available in mia. 
The descriptions follow the following pattern: 

\begin{description}
\item [Plugin:]\emph{the name by which the plugin is selected on the command
line}
\item [Description:] A description of the filter
\item [Parameters:] a list of the plugin parameters

\plugtabstart
parameter name& parameter type& description of the parameter & default value\\\hline
\end{tabular}
\end{description}

\subsection{2D image filters}
\label{sec:2dfilters}

In this section the following notation is used: $g$ indicates the source image,  
   $\hat{f}$ is the output image, and $S_{xy}$ is the set of pixels that is covered by the filter mask.
For an in-depth discussion of most of the filters presented here, see, e.g., \citet{gonzales02:dip}.


\subsubsection*{Convesion of pixel types}
\begin{description}
\item [Plugin:]convert
\item [Description:] Pixel conversion filter
\item [Conversions:] Mapping methods 
\begin{description}
\item [copy:] pixel values are copied, values outside of the target range are clipped
\item [linear:] pixel values are transformed according to $x \rightarrow a \times x + b$ , 
                 values outside of the target range are clipped
\item [opt:] The range of the input pixels \emph{found in the input image} 
		is mapped to the range of the output type. 
\item [range:] The range of the input pixel \emph{type} is mapped to the range of the output type. 
\end{description}
\item [Parameters:] a, b, map, repn

\plugtabstart
a&  float&  scaling for linear mapping & 1.0\\\hline
b&  float&  shift of linear mapping    & 0.0\\\hline
map& string & type of mapping (copy|linear|opt|range) & opt\\\hline
repn & string & target pixel type ( sbyte | ubyte | sshort | ushort |
				sint | uint | slong | ulong | float | double) & ubyte \\\hline
\end{tabular}
\end{description}

\subsubsection*{Downscale Filter}

\begin{description}
\item [Plugin:] downscale
\item [Description:] An filter that uses a seperable kernel to filter a 2D image. 
	For available kernels see Section \ref{sec:spacialkern}
\item [Parameters:] bx, by, kernel

\plugtabstart 
bx & int &  blocksize in x direction  $\in [1,2147483647]$ & 1 \\\hline
by & int &  blocksize in y direction  $\in [1,2147483647]$ & 1 \\\hline
kernel & string &   smoothing filter kernel type to be applied  & gauss \\\hline
\end{tabular}
\end{description}



\subsubsection*{Intensity based bandpass filter}
\begin{description}
\item [Plugin:]bandpass
\item [Description:]An intensity band pass filter - all pixels with intensities
outside the given range are set to zero, all other pixels remain the
same ($x:=x\in [min,max]\:?\: x\::\:0$
\item [Parameters:] min, max

\plugtabstart
min&  float&  lower bound of the bandpass range &0\\\hline
max&  float&  upper bound of the bandpass range &3.40282e+38\\\hline
\end{tabular}
\end{description}

\subsubsection*{Mean Least Variance}
\begin{description}
\item [Plugin:]\emph{mlv}
\item [Description:] an edge preserving and enhancing filter \cite{schulze94morphologybased}
\item [Parameters:] w

\plugtabstart
w & integer & filter width parameter &1\\\hline
\end{tabular}
\end{description}


\subsubsection*{Median Filter}
\begin{description}
\item [Plugin:]\emph{median}
\item [Description:] the median filter, brute force inpmementation 
\item [Parameters:] w

\plugtabstart
w & integer & filter width parameter & 1\\
\hline
\end{tabular}
\end{description}


\subsubsection*{Seperable Convolution Filter}

\begin{description}
\item [Plugin:] sepconv
\item [Description:] An filter that uses a seperable kernel to filter a 2D image. 
	For available kernels see Section \ref{sec:spacialkern}
\item [Parameters:] kx, ky

\plugtabstart
kx&  string &  filter kernel applied in the x-direction & gauss:w=1 \\\hline
ky&  string &  filter kernel applied in the y-direction & gauss:w=1 \\\hline
\end{tabular}
\end{description}

%
%
%

\subsection{3D Filter}
\label{sec:3dfilters}

\subsubsection*{Binarize Filter}
\begin{description}
\item [Plugin:]binarize
\item [Description:]An binarize filter - all pixels with intensities
outside the given range are set to zero, all other pixels are set to one
same ($x:=x\in [min,max]\:?\: x\::\:0$
\item [Parameters:] min, max

\plugtabstart
min&  float&  lower bound of the intensity range &0\\\hline
max&  float&  upper bound of the intensity range &3.40282e+38\\\hline
\end{tabular}
\end{description}


\subsubsection*{Bandpass Intensity Filter}
\begin{description}
\item [Plugin:]bandpass
\item [Description:]An intensity band pass filter - all pixels with intensities
outside the given range are set to zero, all other pixels remain the
same ($x:=x\in [min,max]\:?\: x\::\:0$
\item [Parameters:] min, max

\plugtabstart
min&  float&  lower bound of the bandpass range &0\\\hline
max&  float&  upper bound of the bandpass range &3.40282e+38\\\hline
\end{tabular}
\end{description}

\subsubsection*{Convesion of pixel types}
\begin{description}
\item [Plugin:]convert
\item [Description:] Pixel conversion filter
\item [Conversions:] Mapping methods 
\begin{description}
\item [copy:] pixel values are copied, values outside of the target range are clipped
\item [linear:] pixel values are transformed according to $x \rightarrow a \times x + b$ , 
                 values outside of the target range are clipped
\item [opt:] The range of the input pixels \emph{found in the input image} 
		is mapped to the range of the output type. 
\item [range:] The range of the input pixel \emph{type} is mapped to the range of the output type. 
\end{description}
\item [Parameters:] a, b, map, repn

\plugtabstart
a&  float&  scaling for linear mapping & 1.0\\\hline
b&  float&  shift of linear mapping    & 0.0\\\hline
map& string & type of mapping (copy|linear|opt|range) & opt\\\hline
repn & string & target pixel type ( sbyte | ubyte | sshort | ushort |
				sint | uint | slong | ulong | float | double) & ubyte \\\hline
\end{tabular}
\end{description}


\subsubsection*{Downscale Filter}

\begin{description}
\item [Plugin:] downscale
\item [Description:] An filter that uses a seperable kernel to filter a 3D image. 
	For available kernels see Section \ref{sec:spacialkern}
\item [Parameters:] bx, by, bz, kernel

\plugtabstart 
bx & int &  blocksize in x direction  $\in [1,2147483647]$ & 1 \\\hline
by & int &  blocksize in y direction  $\in [1,2147483647]$ & 1 \\\hline
bz & int &  blocksize in z direction  $\in [1,2147483647]$ & 1 \\\hline
kernel & string &   smoothing filter kernel type to be applied  & gauss \\\hline
\end{tabular}
\end{description}


\subsubsection*{Gaussian Smoothing}
\begin{description}
\item [Plugin:]gauss
\item [Description:] isotropic spacial gaussian smoothing filter 
\item [Parameters:] w

\plugtabstart
w&  int&  filter width parameter [0,2147483647] & 1\\\hline
\end{tabular}
\end{description}

\subsubsection*{Labeling}
\begin{description}
\item [Plugin:]label
\item [Description:] Labeling of connected components in binary images
\item [Remark:] Input image must be binary
\item [Parameters:] n

\plugtabstart
n&  string&  neighborhood mask (see Section \ref{sec:shapes3d}) & n6\\\hline
\end{tabular}
\end{description}


\subsubsection*{Morphological Erode, Dilate, Open, Close}
\begin{description}
\item [Plugin:]erode, dilate, open, close
\item [Description:] the respective morphological filters
\item [Parameters:] hint, shape

\plugtabstart
shape&  string &  structuring element (see Section \ref{sec:shapes3d}) & [sphere:r=1]\\\hline
hint & string & hint to dominant (binary) image content (black|white) & black \\\hline
\end{tabular}
\end{description}


\subsubsection*{Median Filter}
\begin{description}
\item [Plugin:]median
\item [Description:] isotropic median filter 
\item [Parameters:] w

\plugtabstart
w&  int&  filter width parameter $\in[0,2147483647]$ & 1\\\hline
\end{tabular}
\end{description}

\subsubsection*{Mean Least Variance}
\begin{description}
\item [Plugin:]mlv
\item [Description:] isotropic mean least variance filter 
\item [Parameters:] w

\plugtabstart
w&  int&  filter width parameter $\in [0,2147483647]$ & 1\\\hline
\end{tabular}
\end{description}

\subsubsection*{Separable Convolution Filter}

\begin{description}
\item [Plugin:] sepconv
\item [Description:] An filter that uses a seperable kernel to filter a 3D image. 
	For available kernels see Section \ref{sec:spacialkern}
\item [Parameters:] kx, ky, kz

\plugtabstart
kx&  string &  filter kernel applied in the x-direction & gauss:w=1 \\\hline
ky&  string &  filter kernel applied in the y-direction & gauss:w=1 \\\hline
kz&  string &  filter kernel applied in the z-direction & gauss:w=1 \\\hline
\end{tabular}
\end{description}


\subsection{3D Shapes}
\label{sec:shapes3d}

\subsubsection*{6n Neighborhood }

\begin{description}
\item [Plugin:] n6
\item [Description:] a shape object describing the 6n voxel neighbourhood.
\item [Parameters:] 
\end{description}

\subsubsection*{18n Neighborhood }

\begin{description}
\item [Plugin:] n18
\item [Description:] a shape object describing the 18n voxel neighbourhood.
\item [Parameters:] 
\end{description}

\subsubsection*{26n Neighborhood }

\begin{description}
\item [Plugin:] n26
\item [Description:] a shape object describing the 26n voxel neighbourhood.
\item [Parameters:] 
\end{description}

\subsubsection*{6n Neighborhood }

\begin{description}
\item [Plugin:] n6
\item [Description:] a shape object describing the 6n voxel neighbourhood.
\item [Parameters:] 
\end{description}

\subsubsection*{Sphere}

\begin{description}
\item [Plugin:] sphere
\item [Description:] a shape object describing filled sphere
\item [Parameters:] r
\end{description}
\plugtabstart
r&  float  & radius of the sphere & 2 \\\hline
\end{tabular}

\subsection{Spacial Filter Kernels}
\label{sec:spacialkern}

\subsubsection*{Gaussian Convolution Kernel}

\begin{description}
\item [Plugin:] gauss
\item [Description:] The kernel of a spacial Gaussien filter 
\item [Parameters:] w

\plugtabstart
w&  int  & half width of the kernel, actual width = $2 * w + 1$ & 1 \\\hline
\end{tabular}
\end{description}

\subsection{3D Image Combination}
\label{sec:3dcombiners}

\subsubsection*{Label X-reference map}

\begin{description}
\item [Plugin:] labelxmap
\item [Description:] Label cross referencing 
\item [Parameters:] \emph{none}
\end{description}


\subsection{Non-linear image registration}
\label{sec:nrreg}

The following plug-ins are available for both, 2D and 3D implementations of the non-rogid registration algorithms. 

\subsubsection{Cost functions}

\subsubsection*{Sum of Squared Differences}

\begin{description}
\item [Plugin:] ssd
\item [Description:] Sum of Squared Differenes: $F_\text{cost} := \int_\Omega S(x) - R(x) dx$. 
\item [Parameters:] (none)
\end{description}

\subsubsection{Registration models}

These plug-ins provide the solver for the inner loop of the registration, and correspond to the general mathematical model 
  of the registration.  

\subsubsection*{navier}

\begin{description}
\item [Plugin:] navier
\item [Description:] Image registration based on the reduced form of the Navier-Stokes-Equation. 
                       Proviedes linera elastic and fluid dynamic registration without volume preservation. 
\item [Parameters:] epsilon, iter, lambda, mu, omega
\end{description}
\plugtabstart 
epsilon & float &  blocksize in x direction  $\in [1e-6, 0.1]$ & 1e-04 \\\hline
iter & int &  blocksize in y direction  $\in [10, 10000]$ & 100 \\\hline
lambda & float &  isotropic compression $\in [0, \infty]$ & 1.0 \\\hline
mu & float &    isotropic compliance   $\in [0, \infty]$ & 1.0 \\\hline
omega & float &   relaxation parameter $\in [0.1, 10]$ & 1.0 \\\hline
\end{tabular}

\subsubsection*{naviera}

\begin{description}
\item [Plugin:] naviera
\item [Description:] Image registration based on the reduced form of the Navier-Stokes-Equation, adaptive, optimized solver. 
                      Proviedes linera elastic and fluid dynamic registration without volume preservation. 
\item [Parameters:] epsilon, iter, lambda, mu
\end{description}
\plugtabstart 
epsilon & float &  blocksize in x direction  $\in [1e-6, 0.1]$ & 1e-04 \\\hline
iter & int &  blocksize in y direction  $\in [10, 10000]$ & 100 \\\hline
lambda & float &  isotropic compression $\in [0, \infty]$ & 1.0 \\\hline
mu & float &    isotropic compliance   $\in [0, \infty]$ & 1.0 \\\hline
\end{tabular}

\subsubsection*{navierasse}
This plug-in is currently only available for 3D, and if the software was compiled with the GCC compiler or the Intel Compiler

\begin{description}
\item [Plugin:] navierasse
\item [Description:] Image registration based on the reduced form of the Navier-Stokes-Equation, adaptive, optimized solver. 
		      Supports SSE on the Intel platform and AltiVec on PowerPC. 
                      Proviedes linera elastic and fluid dynamic registration without volume preservation. 
\item [Parameters:] epsilon, iter, lambda, mu
\end{description}
\plugtabstart 
epsilon & float &  blocksize in x direction  $\in [1e-6, 0.1]$ & 1e-04 \\\hline
iter & int &  blocksize in y direction  $\in [10, 10000]$ & 100 \\\hline
lambda & float &  isotropic compression $\in [0, \infty]$ & 1.0 \\\hline
mu & float &    isotropic compliance   $\in [0, \infty]$ & 1.0 \\\hline
\end{tabular}

\subsubsection{Timestep models}

These plug-ins provide the time step evaluation in the outer iteration of the registration. 

	
\subsubsection*{Direct evaulation}

\begin{description}
\item [Plugin:] direct
\item [Description:] Provides a ``direct'' time stepping - in the Navier-based models this corrensponds to a linear elastic registration. 
\item [Parameters:] min,max
\end{description}

\plugtabstart 
min & float &  minimum time step $\in [0.001, \infty]$ & 0.1 \\\hline
max & float &  maximum time step $\in [0.002, \infty]$ & 2.0 \\\hline
\end{tabular}

\subsubsection*{Fluid Dynamics}

\begin{description}
\item [Plugin:] fluid
\item [Description:] Provides time stepping that corrensponds to a fluid dynamic model in the Navier-based registration model. 
\item [Parameters:] min,max
\end{description}

\plugtabstart 
min & float &  minimum time step $\in [0.001, \infty]$ & 0.1 \\\hline
max & float &  maximum time step $\in [0.002, \infty]$ & 2.0 \\\hline
\end{tabular}






\bibliographystyle{plainnat}
\cleardoublepage\addcontentsline{toc}{chapter}{\bibname}
\bibliography{userguide}

\end{document}
