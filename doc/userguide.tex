\documentclass[english, 10pt, a4paper,headsepline,openany]{scrbook}
\usepackage[T1]{fontenc}
\usepackage[latin1]{inputenc}
\usepackage[left=2cm,right=1cm,top=2cm,bottom=2cm,twoside]{geometry}
\usepackage{array}
\usepackage{amsmath}
\usepackage{amssymb}
\usepackage[numbers]{natbib}
\usepackage{listings}
\usepackage{color}
\usepackage{graphics}
\usepackage{nonfloat}
\usepackage{babel}
\usepackage{hyperref}

\include{version}
\newcommand{\vx}{\ensuremath{\mathbf{x}} }
\newcommand{\vu}{\ensuremath{\mathbf{u}} }
\newcommand{\n}{\ensuremath{\mathbf{n}} }
\newcommand{\R}{\ensuremath{\mathbb{R}} }
\newcommand{\curl}[1]{\nabla \times #1} % for curl

\newcommand\tabstart{\begin{tabular}{|p{0.12\columnwidth}|
					     p{0.04\columnwidth}|
                                             p{0.1\columnwidth}|
                                             p{0.449\columnwidth}|
                                             p{0.1\columnwidth}|}
                          \hline}
\newcommand\tabend{\end{tabular}}

\newcommand\optinfile{--in-file  & -i & string & input file name & (none) \\\hline}
\newcommand\optreffile{--ref-file  & -r & string & reference file name & (none) \\\hline}
\newcommand\optoutfile{--out-file & -o & string & output file name & (none) \\\hline}
\newcommand\optoutbase{--out-base & -o & string & output file name base - a number and the type name will be added automatically& (none) \\\hline}
\newcommand\opttypetwod{--type      & -t & string & output image file type (see Table \ref{tab:2dformats})  & png \\\hline}
\newcommand\opttypetwodb{--type      & -t & string & output image file type (see Table \ref{tab:2dformats})  & exr \\\hline}
\newcommand\opttypethreed{--type      & -t & string & output image file type (see Section \ref{tab:3dformats})  & vista \\\hline}
\newcommand\opthelpplugin{--help-plugins & & & print out all plugins and a short description & \\\hline}

\newcommand\plugtabstart{\begin{tabular}{|p{0.1\columnwidth}|p{0.1\columnwidth}|p{0.5\columnwidth}|p{0.1\columnwidth}|}\hline
Name & Type & Description & Default \\\hline} 
\newcommand\plugtabend{\hline\end{tabular}}

\newcommand\miamex{{\tt miamex} }
\newcommand{\tw}[1]{{\tt #1}}


%\makeatletter
\definecolor{listinggray}{gray}{0.9}
\lstset{backgroundcolor=\color{listinggray}}


\makeatother

\begin{document}

\vfill{}
\title{MIA User and Programming Guide \\Software Version: \miaversion}
\vfill{}


\author{Gert Wollny}

\maketitle

\section*{Preface}

This is the MIA User and Programming Guide. 
It is dedicated how to use the tools provided by MIA and how to develop software based on the infrastructure provided by MIA.


\section*{License}

Copyright (c) Leipzig, Madrid 2004-2011
Permission is granted to copy, distribute and/or modify this document under the terms of the GNU Free Documentation License, Version 1.1
  or any later version published by the Free Software Foundation; with no Invariant Sections, 
  with no Front-Cover Texts and with no Back-Cover Texts. 
A copy of the license is available at http://www.gnu.org/copyleft/fdl.html

\section*{Changes}

\begin{center}
\begin{tabular}{|c|c|}
\hline 
Date  & Description of changes\\
\hline
\hline 
01/10/2007 & First Public Release  \\
01/03/2007 & Second Beta \\
13/06/2007 & First Beta  \\
\hline
\end{tabular}
\end{center}

\tableofcontents{}

\pagestyle{headings}


\chapter{Introduction}

This is the MIA Programming Guide. This document describes, how mia can be installed and used to 
  write software. 

This document is maintained by Gert Wollny <gw.fossdev@gmail.com>. 
Additions, modifications or corrections may be mailed there for inclusion in the next release. 

\section{Installation}

MIA has been tested on GNU/Linux (x86,amd64) but should be installable on any Posix compatible platform. 


\subsection{UNIX, Linux, Mac OS X} 

In order to use MIA, on these platforms the software is best installed from source code. 
To do so, your software environment has to meet the following pre-requisites: 

\begin{enumerate}
\item You need a ANSI-compatible C++ compiler that supports {\bf auto} from the upcoming 
   C++0x standard - GNU g++ (>=4.4) (http://gcc.gnu.org) is known to work. 
\item CMake (http://www.cmake.org) 
\item The BOOST >= 1.40 (http://www.boost.org) library
\item gsl (http://) The GNU Scientific Library 
\end{enumerate}

Additionally, the following packages add to the functionallity of the software: 

\begin{description}
\item [OpenEXR:] A HDR Image Library that supports 32 bit and floating point valued images (http://www.openexr.org)
\item [TIFF:]    The Tagged Image File Format (http://www.remotesensing.org/libtiff/libtiff.html)
\item [PNG:]     Portable Network Graphics (http://www.libpng.org)
\item [DCMTK:]   DICOM image IO (partial support) 
\item [nlopt:]   Nonlinear optimizers library 
\item [xml++:]   Input/output of XML files, support for certain segmentation features
\item [it++:]    Signal processing library (http://itpp.sourceforge.net) 
\end{description}


If all of the above pre-requisites are met, the installation of mia boils down to:

\lstset{language=bash}
\begin{lstlisting}
  tar -zxvf mia-2.0..tgz
  mkdir build-mia
  cd build-mia
  cmake -DCMAKE_INSTALL_PREFIX=<path to install> ../mia-2.0.0
  make
  make install
\end{lstlisting}
In order to test some of the components, you may run ``make test''. 

\subsection{MS Windows} 

Currently broken 

\chapter{User Guide}

\section{Command Line Tools}

This part of the user guide fucuses on using the command line interface of the MIA programs. 
Although it is possible, to run the programs from the MS Windows command interpreter, installing 
  a UNIX-like environment like Cygwin or MingW is highly advised. 
The examples given below will all use the Bash shell script syntax. 

\subsection{Converting raw data to images and volumes}

MIA doen't support the processing of raw data files. 
Instead it provides the tools to convert such data to certain file formats that carry size and data type information. 
Two command are provided: \texttt{mia-raw2image} for conversion of 2D data and \texttt{mia-raw2volume} for conversion of 3D data. 
Some storage formats also may provide information about pixel/voxel size. 

To convert a raw 2D array of size 120x130 of double floating point values (64 bit) , stored as big endian, and 
  with a pixel size of $2.0 x 2.0$ units to a PNG image, the following command may be issued: 
\lstset{language=bash}
\begin{lstlisting}
mia-raw2image -i input.raw --big-endian -o output.png -r double \
	-s <120,130> -f <2.0,2.0> 
\end{lstlisting}

\noindent 
In order to convert a raw 256x256x128 3D data array of unsingned short values (16 bit), given as low endian, with a voxel size of $1.0 x 1.0 x 2.0$ 
  to the VISTA file format issue the following command: 
\begin{lstlisting}
mia-raw2volume -i input.raw  -o output.v -r ushort \
	-s <256,256,128> -f <1.0,1.0,2.0>
\end{lstlisting}

\noindent 
The number of supported image file formats depends on the compiled in support - see chapter \ref{ch:prog} for details. 

\subsection{Filtering images}

MIA provides a set of standard 2D and 3D filters. 
These filters can easily combined  to a filter chain. 
For example in order to filter a 2D image \texttt{image.png} first with a Gaussian of the order 3 (filter width = 2 * 3 + 1), and then with a 
  median least square filter of the order 2, issue the command 
\begin{lstlisting}
mia-2dimagefilter -i image.png -o filtered.png gauss:w=3 mlv:w=2
\end{lstlisting}

Morphological operations, line erosion, dilation, open, and close are currently only supported for 3D images. 
To apply an openin operation to a volume data set \texttt{volume.v}, by using a filled sphere of radius 3 as structuring element, 
  call: 
\begin{lstlisting}
mia-2dimagefilter -i volume.v -o morphed.v open:shape=[sphere:r=3]
\end{lstlisting}

The available filters and structuring elements are described in chapter \ref{ch:plugins}. 
Given the plug-in based architecture, more filters can easyly ba added.
Datails are described in subsection  chapter \ref{ch:filterplugin} (FIXME:still empty).

\subsection{Non-linear image registration and deformation}

MIA implements fluid dynamic and linear elastic registration by minimising a voxel based measure. 
As a result a vector field is evaluated that describes a probable course of change the given input images and the appllied registration model. 

For a non-linear fluid dynamics based registration of the floating image \texttt{source.v} to the reference image \texttt{reference.v}, by using the 
 adaptive solver algorithm \cite{wollny02comput} with a maximum of 40 iterations and a multi-grid start size of 16x16x16, call 

\begin{lstlisting}
mia-3dnrreg -i source.v -r reference.v -m naviera:iter=40 -m 16 -o field.v 
\end{lstlisting}

\noindent 
In order to apply the deformation to an input image, using the omoms3 interpolation filter, do 
\begin{lstlisting}
mia-3ddeform -i source.v -t field.v -o deformed.v -p omoms3
\end{lstlisting}

\noindent 
More options to the image registration are described in Subsections \ref{sec:reg2d}, \ref{sec:reg3d}, and \ref{sec:nrreg}






%%\section{IDL Interface User Guide}

If MIA is installed into \tw{<MIA-ROOT>} add \tw{<MIA-ROOT>/lib/mia-2.0/plugins/idl} to the \tw{IDL\_DLM} search path. 
After doing so, IDL needs to be restarted. 

\subsection{Supported Functions}

All available functions of the MIA library are imported into IDL by using the DLM interface provided. 
Calling 

\lstset{language=bash}
\begin{lstlisting}
FUNCTION test_echo
in = 10
out = 20
;---------------------------------------------------------
out = MIA_ECHO(in)
;---------------------------------------------------------
IF ( in NE out ) THEN RETURN, 0
RETURN, 1
END
\end{lstlisting}

\noindent 
Supported functions are  \tw{MIA\_ECHO}, \tw{MIA\_DEBUG}, \tw{MIA\_DEFORM2D}, \tw{MIA\_DEFORM3D}, \tw{MIA\_FILTER2D}, \tw{MIA\_FILTER3D}, \tw{MIA\_NRREG2D}, 
  \tw{MIA\_NRREG3D}, and \tw{MIA\_READ2DIMAGE}. 

\section{Function description}

\subsubsection*{Set debug level}

\begin{description}
\item [Function:] MIA\_DEBUG
\item [Description:] set the verbosity of the run algorithm.
\item [Parameters:] verbosity given as string: fatal, error, fail, warning, message, debug (if mia is build without -DNDEBUG)
\item [Example:] Set debug level to 'message'
\begin{lstlisting}
  dummy = MIA_DEBUG('message')
\end{lstlisting}
\end{description}


\subsubsection*{Test function}

\begin{description}
\item [Function:] MIA\_ECHO
\item [Description:] returns the input parameter
\item [Parameters:] an abitrary parameter 
\item [Example:] Echo the number 10 
\begin{lstlisting}
  result = MIA_ECHO(10)
\end{lstlisting}
\end{description}

\subsubsection*{2D/3D image filters}

\begin{description}
\item [Function:] MIA\_FILTER2D, MIA\_FILTER2D
\item [Description:] An 2D/3D image filter as described in section \ref{sec:2dfilters} and \ref{sec:3dfilters}, respectively.
\item [Parameters:] filter, image
\item [Example:] Filtering an 2D image \tw{image} with a median filter of width 3 and store the result in \tw{filtered}.
\begin{lstlisting}
  filtered = MIA_FILTER2D('median:w=3', image)
\end{lstlisting}
\end{description}

\subsubsection*{2D/3D non-rigid image registration}

\begin{description}
\item [Function:] MIA\_NRREG2D, MIA\_NRREG3D
\item [Description:] An 2D or 3D image registration algorithm, fluid dynamics and elastic
\item [Parameters:] src, ref, method, soriter, cost, epsilon, startsize, niter

\begin{tabular}{ll}
src & source (floating) image \\
ref & reference image \\
model & registration model (navier|naviera)\\
timestep & timestep (direct|fluid) \\
cost & cost function \\
epsilon & registration stopping criterion \\
startsize & multigrid start size \\
niter & number of external iterations  \\
\end{tabular}
\item [Example:] running a fluid synamic registration of a pair of 2D images, src and ref, using the sum of 
  squared differences as criterion. The call is executed like follows:
\begin{lstlisting}
deform_field = MIA_NRREG2D(src, ref, 'naviera:iter=40', 'fluid', 'ssd', 0.001, 16, 100)
\end{lstlisting}
\end{description}

\subsubsection*{2D/3D image deformation}

\begin{description}
\item [Function:] MIA\_DEFORM2D, MIA\_DEFORM3D
\item [Description:] An 2D or 3D image deformation, Given a deformation field $\vu$ the deformed image is
   evaulated from the source image $S(\vx)$ according to $D(\vx) := S(\vx - \vu(\vx))$. 
\item [Parameters:] src, field, interp
\item [Returns:] the deformed image

\begin{tabular}{ll}
src & source image  to be deformed \\
field & deformation field $\vu$ \\
interp & interpolator: nn (neares neighbor), tri (bi/tri-linear), bspline3, bspline4, bspline5, omoms3\\
\end{tabular}
\item [Example 1:] deforming a 3d image by using an input field and the tri-linear interpolator
\begin{lstlisting}
    defo = MIA_DEFORM3D(src_img, field, 'bspline3')
\end{lstlisting}
\end{description}


\subsubsection*{2D image reader}

\begin{description}
\item [Function:] MIA\_READ2DIMAGE
\item [Description:] Uses the interface of the MIA library to load an image
\item [Parameters:] name
\item [Returns:] the image

\begin{tabular}{ll}
name  & file name of the image to be loaded \\
\end{tabular}
\item [Example 1:] loading a ``BMP'' file 
\begin{lstlisting}
    image = MIA_READ2DIMAGE('image.bmp')
\end{lstlisting}
\end{description}




%%\section{Matlab Interface User Guide}

If MIA is installed into \tw{<MIA-ROOT>} add \tw{<MIA-ROOT>/lib/mia-2.0/plugins/matlab} to the MATLAB search path. 
 

\subsection{Supported Functions}

All calls to the MIA library are directed through the \miamex function.
The first parameter to \miamex is a command string that describes the requested function, further parameters follow according to the command.
As an example, a random 2D image is filtered by a Gaussian 2D filter of filter parameter 3 wich results in a filter width of 7:

\lstset{language=matlab}
\begin{lstlisting}
image = rand(256, 128) 
filtered_image = miamex('filter2d', 'gauss:w=3',  image) 
\end{lstlisting}

\noindent 
Supported commands are  \tw{debug},  \tw{deform2d}, \tw{deform3d}, \tw{filter2d}, \tw{filter3d}, \tw{reg2d}, \tw{reg3d}. 
For convenience, a set of functions is defined as MATLAB functions:  \tw{miadeform2d}, \tw{miareg2d}, \tw{miadeform3d},  and \tw{miareg3d}. 


\subsection{Image Filters}

\subsubsection*{Set debug level}

\begin{description}
\item [Command:] debug
\item [Description:] set the verbosity of the run algorithm.\footnote{%
	It seems for some reason, on MS Windows this is not output properly. 
}
\item [Parameters:] verbosity: fatal, error, fail, warning, message, (debug)
\item [Example:] Set debug level to 'message'
\begin{lstlisting}
  miamex('debug', 'message')
\end{lstlisting}
\end{description}

\subsubsection*{2D/3D image filters}

\begin{description}
\item [Command:] filter2d, filter 3d
\item [Description:] An 2D/3D image filter as described in section \ref{sec:2dfilters} and \ref{sec:3dfilters}, respectively.
\item [Parameters:] filter, image
\item [Example:] Filtering an 2D image \tw{image} with a median filter of width 3 and store the result in \tw{filtered}.
\begin{lstlisting}
  filtered = miamex('filter2d', 'median:w=3', image)
\end{lstlisting}
\end{description}

\subsubsection*{2D/3D non-rigid image registration}

\begin{description}
\item [Command:] reg2d, reg3d
\item [Convenience:] miareg2d, miareg3d
\item [Description:] An 2D or 3D image registration algorithm, fluid dynamics and elastic
\item [Parameters:] src, ref, method, soriter, cost, epsilon, startsize, niter

\begin{tabular}{ll}
src & source (floating) image \\
ref & reference image \\
soriter & number of iterations for solver \\
cost & cost function \\
epsilon & registration stopping criterion \\
startsize & multigrid start size \\
niter & number of external iterations  \\
\end{tabular}
\item [Example:] running a fluid synamic registration of a pair of 2D images, src and reference, using the sum of 
  squared differences as criterion. The call is executed by using the convenience fucnction
\begin{lstlisting}
deform_field = miareg2d(src, ref, 'fluid', 20, 'ssd', 0.001, 16, 100)
\end{lstlisting}
\end{description}

\subsubsection*{2D/3D image deformation}

\begin{description}
\item [Command:] deform2d, deform3d
\item [Convenience:] miadeform2d, miadeform3d
\item [Description:] An 2D or 3D image deformation, Given a deformation field $\vu$ the deformed image is
   evaulated from the source image $S(\vx)$ according to $D(\vx) := S(\vx - \vu(\vx))$. 
\item [Parameters:] src, field, interp
\item [Returns:] the deformed image

\begin{tabular}{ll}
src & source image  to be deformed \\
field & deformation field $\vu$ \\
interp & interpolator: nn (neares neighbor), tri (bi/tri-linear), bspline3, bspline4, bspline5, omoms3\\
\end{tabular}
\item [Example 1:] deforming a 3d image by using an input field and the tri-linear interpolator
\begin{lstlisting}
deform_image = miamex('deform3d', image3d, field3d, 'tri')
\end{lstlisting}
\item [Example 2:] deforming a 2d image by using an input field and the convenience function
\begin{lstlisting}
deform_image = miadeform2d(image2d, field2d)
\end{lstlisting}
\end{description}







\chapter{Programs}
\label{ch:prog}


This chapter lists all the software tools provided by mia2D. 
The descriptions follow a designated pattern: 

\begin{description}
\item [Program:]\emph{the command line program name}
\item [Description:]A description of the program
\item [Remarks:] additional notes
\item [Options:] list of options

\tabstart
long name & short name & value type & description of the parameter & default value\\
\hline
\tabend

\noindent 
If no value type is given, then the option accepts no further parameter. 
All programs support the following options: 

\tabstart
--help & -? & & Print out some help & \\\hline
--usage & & & Print out a short help & \\\hline
--version & -v & & Print out version information & \\\hline
--copyright & & & Print out copyright information & \\\hline
--verbose & -V & string & set the verbosity of the output out of \{ debug, message, warning, fail, error, fatal\} & error \\\hline
\tabend
\end{description}

Input from and output to a variety of 2D and 3D file formats is implemented. Note, however, only binary and gray scale data is supported.

\begin{table}[h]
\caption{\label{tab:2dformats}2D image formats}
\centering{
\begin{tabular}{|l|l|c|}
\hline
File type description & on the web & suffix \\\hline
OpenEXR HDR image format & http://www.openexr.com/ & exr \\\hline
Portable Network Graphics & http://www.libpng.org & png \\\hline
Tagged Image File format & http://www.remotesensing.org/libtiff/libtiff.html & tif \\\hline
Windows Bitmap & http://www.daubnet.com/formats/BMP.html & bmp \\\hline
\end{tabular}
}	
\end{table}

\begin{table}[h]
\caption{\label{tab:3dformats}3D image formats}
\centering{
\begin{tabular}{|l|l|c|}
\hline
File type description & on the web & suffix \\\hline
INRIA 3D File format &http://foveaproject.free.fr/softwareEng.html & inria  \\\hline
%%Medical Imaging NetCDF & http://www.bic.mni.mcgill.ca/software/minc/ & mnc \\\hline
Sun TAAC Image File Format & & vff \\\hline
Vista File Format & http://www.cs.ubc.ca/nest/lci/vista/vista.html & v \\\hline
\end{tabular}
}
\end{table}

\section{2D Image Conversion}

\begin{description}
\item [Program:]\emph{mia-raw2image}
\item [Description:]A raw data to 2D image image conversion program. 
\item [Options:] $\:$

\tabstart
\optinfile
\optoutfile
\opttypetwod
--big-endian & -b & bool & input data is big-endian & false \\\hline
--repn & -r & string & input pixel type (ubyte|sbyte|short|ushort|int|uint|float|double) & short \\\hline
--scale & -f & vector of float & scale of input voxels FX,FY in real world units & 1.0,1.0 \\\hline
--size & -s & vector of integer & size of input NX,NY & 1,1 \\\hline
\tabend
\item [Example:] Converting \emph{unsigned short} data in low-endian format of size $120\times130$ representing data 
  of pixel size $2mm \times 2mm$ to a tif image:
\begin{lstlisting}
mia-raw2image -i input.raw -o output.tif -r ushort -s '<120,130>' -f '<2.0,2.0>' 
\end{lstlisting}
\end{description}


\section{2D Image Filtering}

\begin{description}
\item [Program:]\emph{mia-2dimagefilter}
\item [Description:]A 2D image filtering program. Supported filters are described in Subsection \ref{sec:filter2d}. 
\item [Options:] $\:$

\tabstart
\optinfile
\optoutfile
\opttypetwod
\opthelpplugin
\tabend
\item [Example:] Filtering an \texttt{input.png} with a median filter and anisotropic filtering and writing the output to \texttt{output.bmp}, 
		thus converting the image to the Microsoft bitmap format after filtering. 
\begin{lstlisting}
mia-2dimagefilter -i input.png -o output.bmp -t bmp \
	median:w=3 aniso:iter=1000,psi=pm1
\end{lstlisting}
\end{description}

\begin{description}
\item [Program:]\emph{mia-2dimagefilterstack}
\item [Description:]A 2D image filtering program that filters a series of images that are numbered consecutively. 
              Supported filters are described in Subsection \ref{ch:plugins}. 
\item [Options:] $\:$

\tabstart
\optinfile
\optoutbase
\opttypetwod
\opthelpplugin
\tabend
\item [Example:] Filtering images of the pattern \texttt{input0000.png} with a median filter and anisotropic 
                 filtering and writing the output to \texttt{outputXXXX.bmp}, thus converting the image to the Microsoft 
                 bitmap format after filtering. 
		 XXXX represents the slice numbers that correspond to the input image slice numbers. 
                 The number of digits corresponds to the number of digits in the input image file names. 
\begin{lstlisting}
mia-2dimagefilterstack -i input0000.png -o output -t bmp \
        median:w=3 aniso:iter=1000,psi=pm1
\end{lstlisting}
\end{description}


\section{2D Non-Linear Image Registration}
\label{sec:reg2d}

\begin{description}
\item [Program:]\emph{mia-2dnrreg}
\item [Description:]voxel based 2D image registration software 
\item [Options:] $\:$

\tabstart
\optinfile
\optreffile
\optoutfile
--def-file & -d & string &  deformed inpout image & (none) \\\hline
--regmodel & -m & string & registration model  & navier \\\hline
--timestep & -t  & string & time step (fluid|direct) & fluid \\\hline
--mgsize  & -s & integer & multigrid start size & 16 \\\hline       
--max-iter & -n & maximum number of iterations & 200  \\\hline
--cost & -c & cost function & ssd             \\\hline
--interpolator & -p  & image interpolator (bspline2|bspline3|bspline4|bspline5|nn|omoms3|tri) & bspline3  \\\hline
--epsilon & -e &  relative accuracy to stop registration at a multi-grid level & 0.01  \\\hline
\tabend
\item [Example:] Registration of two images by using the naviera kernel and standart parameters:
\begin{lstlisting}
   mia-2dnrreg -i src.png -r ref.png -o src-ref-field.vf -m naviera 
\end{lstlisting}
\end{description}


\section{3D Image Conversion}

\begin{description}
\item [Program:]\emph{mia-raw2volume}
\item [Description:]A raw data to 3D image image conversion program. 
\item [Options:] $\:$

\tabstart
\optinfile
\optoutfile
\opttypetwod
--big-endian & -b & bool & input data is big-endian & false \\\hline
--repn & -r & string & input pixel type (ubyte|sbyte|short|ushort|int|uint|float|double) & short \\\hline
--scale & -f & vector of float & scale of input voxels FX,FY,FZ in real world units & <1.0,1.0,1.0> \\\hline
--size & -s & vector of integer & size of input NX,NY,NZ & <1,1,1> \\\hline
\tabend
\item [Example:] Converting \emph{unsigned short} data in low-endian format of size $120\times130\times64$ representing data 
  of pixel size $2mm \times 2mm \times 1mm$ to a tif image:
\begin{lstlisting}
mia-raw2image -i input.raw -o output.tif -r ushort -s '<120,130,64>' -f '<2.0,2.0,1.0>'
\end{lstlisting}
\end{description}


\begin{description}
\item [Program:]\emph{mia-3dimagefilter}
\item [Description:]A 3D image filtering program. Supported filters are described in Subsection \ref{sec:3dfilters}. 
\item [Options:] $\:$

\tabstart
\optinfile
\optoutfile
\opttypetwod
\opthelpplugin
\tabend
\item [Example:] Filtering an \texttt{input.png} with a median filter and anisotropic filtering and writing the output to \texttt{output.bmp}, 
		thus converting the image to the Microsoft bitmap format after filtering. 
\begin{lstlisting}
mia-2dimagefilter -i input.png -o output.bmp -t bmp \
	median:w=3
\end{lstlisting}
\end{description}


\section{3D Non-Linear Image Registration}
\label{sec:reg3d}

\begin{description}
\item [Program:]\emph{mia-2dnrreg}
\item [Description:]voxel based 2D image registration software 
\item [Options:] $\:$

\tabstart
\optinfile
\optreffile
\optoutfile
--def-file & -d & string &  deformed inpout image & (none) \\\hline
--regmodel & -m & string & registration model  & navier \\\hline
--timestep & -t  & string & time step (fluid|direct) & fluid \\\hline
--mgsize  & -s & integer & multigrid start size & 16 \\\hline       
--max-iter & -n & maximum number of iterations & 200  \\\hline
--cost & -c & cost function & ssd             \\\hline
--interpolator & -p  & image interpolator (bspline2|bspline3|bspline4|bspline5|nn|omoms3|tri) & bspline3  \\\hline
--epsilon & -e &  relative accuracy to stop registration at a multi-grid level & 0.01  \\\hline
\tabend
\item [Example:] Registration of two images by using the naviera kernel and standart parameters:
\begin{lstlisting}
   mia-3dnrreg -i src.png -r ref.png -o src-ref-field.vf -m naviera 
\end{lstlisting}
\end{description}


\include{plugins_new}

\chapter{Programming Guide}
\lstset{numbers=left, numberstyle=\small, numbersep=5pt}

This part of the MIA guide decribes, how to use the interfaces provided to write your own softwrae, 
  how to extend the the functionallity of MIA without touchting its core, and, finally, how to 
  change MIA itself. 
\begin{itemize}
\item In chapter \ref{ch:simple} an example is provided, that describes how to scan the command line, load an image, 
  run some given filters on it and store the image. 
\item Chapter \ref{ch:images} focuses images on writing an image filter that can handle various different gray scale pixel formats.
\item Chapter \ref{ch:io} decribes how to teach MIA to read and write additional image formats. 
\item Chapter \ref{ch:filterplugin} will teach you, how to write an image filter plugin. 
\item Chapter \ref{ch:addpluginstype} gives an insight how to create a new class of plug-ins.
\end{itemize}

\section{A simple program}
\label{ch:simple}

This chapter will teach you how to use the MIA-tools to scan command line parameters, read and store images, 
  create filters opbejcts from plug-ins and apply filter chains to the images. 
In summary, the program described here is the \texttt{mia-2dimagefilter} from \texttt{src/2dimagefilter.cc}. 

For convenience the full source code is printed in the listning below, and we will go through it line by line. 
\lstset{language=c++}
\begin{lstlisting}
#include <mia/core.hh>
#include <mia/2d.hh>

NS_MIA_USE; 
using namespace std; 

int main( int argc, const char *argv[] )
{
  string in_filename;
  string out_filename;
  string out_type; 
  bool help_plugins = false; 

  try 
  {
    const C2DFilterPluginHandler::Instance& filter_plugins = 
      C2DFilterPluginHandler::instance();
  
    const C2DImageIOPluginHandler::Instance& imageio = 
      C2DImageIOPluginHandler::instance();
  
    CCmdOptionList options;  
    options.push_back(make_opt( in_filename, "in-file", 'i', 
                      "input image(s) to be filtered", "input", true)); 
    options.push_back(make_opt( out_filename, "out-file", 'o', 
                      "output image(s) that have been filtered", "output", true)); 
    options.push_back(make_opt( out_type, imageio.get_set(), "type", 't',
                      "output file type (if not given deduct from output file name)" , 
                      "image-type"));
    options.push_back(make_opt( help_plugins, "help-plugins", 0, 
                      "give some help about the filter plugins", NULL)); 

    options.parse(argc, argv); 
    vector<const char *> filter_chain = options.get_remaining(); 

    list<C2DFilterPlugin::ProductPtr> filters;
    for (vector<string>::const_iterator i = filter_chain.begin(); 
         i != filter_chain.end(); ++i) {
       C2DFilterPlugin::ProductPtr filter =  filter_plugins.produce(i->c_str()); 
       if (!filter) {
         stringstream error; 
         error << "Filter " << *i << " not found"; 
         throw invalid_argument(error.str());
       }
       filters.push_back(filter);
    }

    C2DImageIOPluginHandler::PData  in_image_list = imageio.load(in_filename);  
    if (!in_image_list || in_image_list->empty())
         throw invalid_argument(string("No images found in ")  + in_filename);

    vector<string>::const_iterator filter_name = filter_chain.begin();
    for (list<C2DFilterPlugin::ProductPtr>::const_iterator f = filters.begin(); 
         f != filters.end(); ++f) {
      for (C2DImageIOPluginHandler::Data::iterator   
         i = in_image_list->begin(); i != in_image_list->end(); ++i)
    
           *i = (*f)->filter(**i);
    }
    if ( !imageio.save(out_type, out_filename, *in_image_list) )
      throw runtime_error(string("unable to save result to ") + out_filename);
    return EXIT_SUCCESS; 

  } 
  catch (const runtime_error &e){
    cerr << argv[0] << " runtime: " << e.what() << endl;
  }
  catch (const invalid_argument &e){
    cerr << argv[0] << " error: " << e.what() << endl;
  }
  catch (const exception& e){
    cerr << argv[0] << " error: " << e.what() << endl;
  }
  catch (...){
    cerr << argv[0] << " unknown exception" << endl;
  }
  return EXIT_FAILURE;
}  
\end{lstlisting}  

\begin{description}
\item [1:] For option parsing include the core of the MIA library. 
\item [2:] For 2D image handling pull in the apropriate declaration. 
\item [4-5:] Using the MIA and STD namespace for short access. 
\item [7:] Start the main function. 
\item [9-12:] Declare some variables, that are needed for the command line parsing, namely the input and output file names, 
    the output file format (if not given the file format is deducted from the output file name), and a flag 
    see if the user requests some help on the supported filter plugins. 
    The values given to these variables will be the default values for the options. 
     As an exception, boolean values are always default to ``false''. 
\item [14-15, 64-77:] The whole code is put into a try catch block, since error handling is done by exceptions. 
\item [16-20:] Define some short cuts for the plugin handlers.
\item [22-31:] Define the options for this program using the variables given above. 
    Silently, in the background are some more options available, that deal with common tasks, such as printing out the general
      help, printing out copyright information, and verbosity level of the output. 
    See chapter \ref{ch:prog} for details. 
\item [33-34:] Parse the command line. 
     For this program, the remaining command line parameters describe the filters to be applied. 
    A command line like 
    \lstset{language=bash,numbers=none}
    \begin{lstlisting}
eva-2dimagefilter -i input.png -o output.png \
    downscale:bx=2,by=2 bandpass:min=20,max=200
    \end{lstlisting}
      will be parsed like follows:
      \begin{itemize}
      \item  \texttt{in\_filename} = ``input.png''
      \item  \texttt{out\_filename} = ``output.png''
      \item  \texttt{filter\_chain} = (downscale:bx=2,by=2, bandpass:min=20,max=200)
      \end{itemize}

\item [36-46:] Create the filter chain by using the filter descriptions from the command line. 
       If the creation of one filter fails, an \texttt{invalid\_argument} exception is thrown, which will terminate the 
       program with an error message. 
\item [48-50:] Load the image(s) from the input file. 
       If no input image is found, throw an \texttt{invalid\_argument} exception that will terminate the program with an error message. 
\item [52-59:] Apply the filters to all input images replacing the input images. 
       Since the images are all wrapped into shared pointers, no memory leaks will be introduced by this code. 
\item [60-61:] Save the filtered image(s) in the output file. If not successfull throw an exception.
\item [62:] Terminate the program with the return value indicating success.
\item [64-77:] Catch all known and unknown exceptions and report.
\item [78:] return signalling failure, because somewhere in the program a exception was thrown that could only be reported 
    in the main function. 
\end{description}

\section{Images and Filtering}

\label{ch:images}

In the following, the basic handling of images is described. 
All the examples will use 2D images. 
If not otherwise noted, 3D images are handled likewise. 

\section{How to create and copy images of a certain image type}

The images supported in MIA  may contain different pixel types, e.g. 1-bit, 8-bit, 16-bit, or even float valued ones. 
MIA uses shared pointers to \texttt{C2DImage}, respectively,  to hold images. 
However, in order to account for the pixel type, at creation time a derivative class of the used pixel type needs to be specified. 
In order to create an 16-bit image with unsigned pixel values with a given \texttt{size}, use:

\begin{lstlisting}
P2DImage image = P2DImage(new C2DUSImage(size)); 
\end{lstlisting}
 
In order to copy such an image, an image copy filter is provided, that can be invoked like follows: 

\begin{lstlisting}
P2DImage image_copy = filter<T2DImage>(FCopyImage(), image); 
\end{lstlisting}
\emph{Remark: a clone method might come in handy ...}

More about these image filters is described in the next section: 

\section{An Image Filter}
\label{sec:filter}

In order to access the image data, its type needs to be known. 
Of course it is possible to use the \texttt{get\_type} method of the \texttt{C2DImage} class and use
  a switch statement in order to cast to the appropriate derived class. 
However, since this needs to be done very often, MIA provides various template functions to handle the most cases
  of image access. 
Namely, functions \texttt{mia::filter} are provided that take a filter functor as argument as well as one image or two images. 
Examples of its usage follow: 
Imagine, an image thresh-holding filter, that takes an image, and sets all pixels above or below given thresholds to zero, and 
  all pixels within the range to one. 
Such a filter would be defined like follows:     

\lstset{language=c++,numbers=left}	
\begin{lstlisting}
#include <mona/core/filter.hh>

struct FThreshold: public TFilter<P2DImage> {
  FThreshold(double min, double max):
    _M_min(min), 
    _M_max(max)
    {
    }
  typename <template T> 
  FThreshold::result_type operator()(const T2DImage<T>& image)const {
    typename T2DImage<T>::const_iterator ii = image.begin(); 
    typename T2DImage<T>::const_iterator ei = image.end();
    
    C2DBitImage *r = new  C2DBitImage(image.get_size(), image.get_attribute_list()); 
    P2DImage result(r); 
    C2DBitImage::iterator ir = r->begin();
    
    while (ii != ei) {
      *ir++ = (*ii >= _M_min && *ii < _M_max); 
      ++ii; 
    }
    return result;     
  }
private: 
  double _M_min; 
  double _M_max; 
); 
\end{lstlisting}

Here, \texttt{TFilter} is a template, that defines the type \texttt{return\_type} that is needed for the call to the actual filter function. 
The constructor of the filter functor \texttt{FThreshold} initialises the filter with the given parameters. 
A \texttt{do\_threshold} function that makes use of this filter operator would look like this:   
  
\begin{lstlisting}
P2DImage do_threshold(double min, double max, const C2DImage& image)
{  
  return  mia::filter<T2DImage>(FThreshold(min,max), image);
}   
\end{lstlisting}

The \texttt{filter<T2DImage>} function takes care of casting the input image depending on its pixel type, and then invokes 
  the operator () of \texttt{FThreshold} and returns its result.
  
\section{Pixel Type Dependant Filtering}
\label{sec:ptdf}

Imaging the situation, when a filter type is only appropriate for certain pixel types, or should behave differently for different
   pixel types. 
Then using template specification comes to the rescue. 
Imagine, for example, in the above example, using the thresh-holding filter on a bit-valued image doesn't make much sense. 
Therefore, we would like to report an error, if the filter is invoked with an bit valued image. 
This can be done  by additionally implementing a specialisation of the operator () of \texttt{FThreshold}: 
\emph{I think another indirection is needed here ...}
\begin{lstlisting}
template <>
FTreshold::result_type FTreshold::operator(const C2DBitImage& image)const 
{
  throw invalid_argument("FTreshold can not be used on bit-valued images"); 
}
\end{lstlisting}
 
The compiler will take care of the rest, and if the user provides a bit valued image the above exception will be raised. 


\section{Extending MIA}
\label{ch:io}

\subsection{Adding a Plugin to store/load data in a new file format}
\label{sec:newfilehandler}

In the following, adding support fo a new 2D image file format is described. 
The base class defining the interface for 2D Image IO is \texttt{C2DImageIOPlugin} and it is 
  declared in the the header file \texttt{<mia/2d/2dimageio.hh>}. 
In order to implement the new IO file format, four abstract virtual functions have to be implemented: 
\begin{itemize}  
\item {\tt do\_add\_suffixes}: Adds the standard suffixes for this file format to the plugin handlers suffic map. 
\item {\tt do\_load}: Implements loading data from the file. 
\item {\tt do\_save}: Implements storing data to the file. 
\item {\tt do\_get\_descr}: Returns a string with a short description of the plugin.
\end{itemize}  
In addition, a constructor has to be defined, in order to name the format. 

The following example code is taken from the Windows BMP iamge format IO plugin which resides in the MIA source tree in \texttt{mia/2d/io/bmp.cc}.
First, the plugin class is declared as a specification of the general 2D plugin class: 

\begin{lstlisting}{BMPIO}
#include <mia/2d/2dimageio.hh>
NS_BEGIN(BMPIO)
NS_MIA_USE

class CBMP2DImageIO: public C2DImageIOPlugin {
public:
  CBMP2DImageIO();
private: 
  void do_add_suffixes(multimap<string, string>& map) const;
  PData do_load(const string& fname) const;
  bool do_save(const string& fname, const Data& data) const;
  const string do_get_descr() const; 
};
\end{lstlisting}

\noindent 
The constructor of the plugin takes care of two things: initialise the plug-in with its apropriate name and set the supported pixel formats. 

\begin{lstlisting}{BMPIO}
CBMP2DImageIO::CBMP2DImageIO():
	C2DImageIOPlugin("bmp")
{
	add_supported_type(it_ushort);
	add_supported_type(it_ubyte); 
	add_supported_type(it_bit);
}
\end{lstlisting}

\noindent 
Since at construction time, nothing is known about the plugin handler, the setup of the suffix-fileformat map has to be done in an extra function: 
\begin{lstlisting}{BMPIO}
void CBMP2DImageIO::do_add_suffixes(multimap<string, string>& map) const
{
	map.insert(pair<string,string>(".bmp", get_name())); 
	map.insert(pair<string,string>(".BMP", get_name())); 
}
\end{lstlisting}


\noindent 
The next function just gives a plain text description of the plug-in, that might be used wehn printing out help. 
\begin{lstlisting}{BMPIO}
const string  CBMP2DImageIO::do_get_descr()const
{
	return string("BMP 2D-image input/output support");
}
\end{lstlisting}

\noindent 
Since some file types (like, e.g. TIFF) provide their own routines for opening the file, this filr oprn code needs to be added here.
After creating the result image vector, format specific code follows to load the image data that is then returned. 
\begin{lstlisting}{BMPIO}
CBMP2DImageIO::PData CBMP2DImageIO::do_load(string const& filename)const
{
  int read = 0; 
  CInputFile f(filename);
  if (!f)
    return PData(); 

  PData result = PData(new C2DImageVector()); 

 
  // File format specific implementation follows
  //  ...
	
  return result; 
}
\end{lstlisting}

\noindent 
In the save function, again the file needts to be opened/created for output, because this might by format specific. 
Saving the image data requires a translation of the pixel type that is best provided by a filter function (see Section \ref{sec:filter}) here 
  given by the CImageSaver class.
\begin{lstlisting}{BMPIO}
bool CBMP2DImageIO::do_save(string const& filename, const C2DImageVector& data) const
{
	cvdebug() << "CBMP2DImageIO::save begin\n"; 
	
	COutputFile f(filename);
	if (!f) {
		cverr() << "CBMP2DImageIO::save:unable to open output file:" << filename << "\n"; 
		return false; 
	}
		
	CImageSaver saver(f); 
	
	for (C2DImageVector::const_iterator iimg = data.begin(); iimg != data.end(); ++iimg)
		filter(saver, **iimg); 
	
	cvdebug() << "CBMP2DImageIO::save end\n"; 
	return true; 
}
\end{lstlisting}

\noindent 
Finally, the plugin interface function needs to be defined. 
This functions needs to be declared \texttt{extern ``C''} to avoid the C++ name mangeling that is compiler dependend. 
Depending on the target platform and compiler, some additional decoration might be necessary that is set elsewhere 
  through the \texttt{EXPORT} define.
\begin{lstlisting}{BMPIO}
extern "C" EXPORT CPluginBase *get_plugin_interface()
{
		return new CBMP2DImageIO;
}
\end{lstlisting}

\noindent 
Above code skeleton provides the basis for all IO plugins. 
For further reading, the source code in the respective ``io'' directories might hold forward. 

\subsection{Writing an Image Filter Plugin}
\label{ch:filterplugin}

\subsection{Adding a new general plugin type}
\label{ch:addpluginstype}

\subsection{Adding new data types}
\label{sec:adddatatypeio}


\bibliographystyle{plainnat}
\cleardoublepage\addcontentsline{toc}{chapter}{\bibname}
\bibliography{userguide}

\end{document}
